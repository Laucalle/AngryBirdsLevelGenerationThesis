%************************************************
\subsection{Motivation}
It is common for AI researchers to turn to games in general as a testing environment for AI techniques. Some of the earliest problems that AI attempted to solve were checkers and chess in the 50s, even before AI was defined and recognized as a field in the Dartmouth workshop in 1956 \cite{nilsson1998artificial}.

What made board games attractive in the first place? They have a rather simple set of rules but winning a game could be a challenging task even for a human brain. It is not surprising that soon video games too drew researchers' attention---along with board games. They offer a wide range of dynamic and competitive elements that resemble real-world problems to some extent.

Of course, this interest works in both ways. Many video games use AI techniques to deliver better experiences, mainly involving NPC behaviour and PCG.

Having this relationship between academia and industry in mind, the IEEE Conference on Computational Intelligence and Games started as a symposium in 2005, and as a conference in 2009. It brings professionals from both fields together to discuss the latest advances in AI and Computational Intelligence and how they apply to games.\cite{ieee-cig}

Among other events, the conference hosts competitions. In 2018, most proposed competitions are centred around playing AI agents for specific games or genres. One of the exceptions is the \textit{3rd Angry Birds Level Generation Competition}. Participants must build computer programs that are able to generate levels for the\textit{ Angry Birds} game.

\textit{Angry Birds} is a mobile game by Finnish company Rovio Entertainment Corporation\cite{angry-birds}, first launched in 2009. The game was a huge success in the \textit{AppStore}---being the most downloaded mobile game in history. Having been ported to several other platforms since, it has an animated movie and way too many \textit{spin offs}---around seventeen. In the game, green pigs have stolen the birds' eggs, and they proceed to rescue them. The pigs have built a variety of defensive structures made out of blocks, so the player has to fire the birds from a slingshot---apparently they never learnt to fly---so the structure is destabilized.

However, what is interesting to us is not its wide success, but how heavily the game relies in gravity to create interesting puzzles. The main challenge from the PCG perspective is to build stable structures that are robust enough to take more than a single shot before crumbling to the ground.

% The main challenge [from the PCG perspective (or similar)] is to build stable structures.. - Mario
 

The ultimate objective of this project is to build a program capable of creating levels for \textit{Angry Birds}. This by itself is too vague to convey the desired approach for the project, so we can break it down to the following, more concise, objectives:

\begin{itemize}
	\item Explore the expressiveness and variability of SBPCG using evolutionary algorithms
	\item Adapt the game code to extract data from execution, with the aim of evaluating the levels.
	% Adapt the [game code], to extract (readers may think about the actual game) - Mario 
	\item Produce stable structures under gravity.
	\item Place other elements on the structures to complete the levels.
\end{itemize} 


