%************************************************

\subsection{What is Procedural Content Generation}
The definition of Procedural Content Generation (PCG) has been broadly discussed but there is no agreement. We have plenty of examples of what \textit{is} and what \textit{is not} PCG, but every definition struggles to cover all cases, either being too inclusive or too exclusive. The one we choose here balances well between the two, defining PCG as \textit{the algorithmical creation of game content with limited or indirect user input} \cite{togelius2011procedural}.

% You can define PCG, NPC, AI the first time they are mentioned

Although PCG often uses Artificial Intelligence (AI) techniques, this definition does not include all uses of it in games. We do not consider Non Playable Character (NPC) behaviour or AI playing agents as content, thus they are not PCG either. Aesthetic elements, game rules, levels, items, stories and characters among others are considered content in this definition.


Note that neither computers nor video games are mentioned in the definition. In fact, PCG has its roots in analogical games. This may conflict with the \textit{limited or indirect user input}, but it is reasonable to assume that following a detailed set of instructions---even if it is done by a human---is not \textit{input}. The underlying concepts used much earlier by non-digital games still prevail in modern video games. Using an algorithm to assemble pre-designed pieces is a common technique in tabletop roleplaying guides---where the algorithm usually consists in several dice rolls---such as \textit{Dungeon \& Dragons}. It is not surprising that one of the early adaptations of PCG to digital platforms aimed to generate monsters and dungeons for physical games.\cite{smith2015analog}

% In a conference paper, we shouldn't probably go down to the level of subsubsection (or, for that matter, subsection) - JJ
% DONE - Laura

There are many PCG methods and it is necessary to look at some traits that characterize and differentiate them from each other: \cite{togelius2016introduction}

\begin{itemize}
	\item \textit{Online/offline}: In online generation, the PCG occurs during the game session, while the user is playing the game. If it is done before the game session or during development, we have offline generation. 
	\item \textit{Necessary/optional}: Procedurally generated content may be part of the essential structure of the game or can be additional to the game experience and can be discarded. In the first one the content is \textit{necessary} and needs to be correct while the latter is \textit{optional}.
	\item \textit{Degree and dimensions of control}: As any other algorithm, a PCG method can have a number of parameters which affect the output. If it uses random numbers, the seed is one of them.
	\item \textit{Generic/adaptative}: Adaptative generation takes into account player behaviour while generic does not. Although there are exceptions, most commercial games choose generic over adaptative.
	\item \textit{Stochastic/deterministic}: Deterministic PCG will produce the same output given the same input, in contrast to stochastic generation which is not as easy to replicate. 
	\item \textit{Constructive/generate-and-test}: Generate-and-test produces potentially correct solutions that are tested and adjusted in each iteration before giving the actual output. Constructive methods build partial solutions and add on them.
	\item \textit{Automatic/mixed authorship}: PCG can be used as an assisting tool for designers, whether the output is used as a base or as an interactive process. Then we talk about mixed authorship, as opposed to the automatic generation where the designer does not take part. 
\end{itemize}

\subsubsection{Search-based Procedural Content Generation}

A special kind of \textit{generate-and-test} approach to PCG is Search-based Procedural Content Generation (SBPCG), which is usually---but not always---tackled with Evolutionary Algorithms. The problems faced by SBPCG are not very far from those encountered in Evolutionary problems. 

The test function does not determine if a solution is valid, but it grades how good the solution is. This is often called \textit{fitness function}. In SBPCG, how to evaluate the quality of a solution has no straightforward answer. It requires formalizing as a \textit{fitness function} how much fun, exciting or engaging certain content is, which are usually based in tricky assumptions.

% It requires formalizing as a \textit{fitness function} how much fun, exciting or engaging certain content is.  
% DONE - Laura

There are three main classes of \textit{fitness functions} in SBPCG\cite{togelius2010search}:

\begin{itemize}
	\item \textit{Direct Fitness Function} where certain features are extracted from the generated content and directly mapped to a fitness value. Those features must be easily measurable.
	\item \textit{Simulation-Based Fitness Function} where an agent plays through some part of the game that involves the generated content. The fitness is calculated using features from the agent's gameplay. 
	% where an agent plays through some part 
	% DONE - Laura
	
	\item \textit{Interactive Fitness Function} where the fitness value is obtained from the player, whether it be explicitly by asking them or implicitly by measuring certain responses to the game. 
\end{itemize}

% This should go _after_ the definition, and probably before Motivation. - JJ
% DONE - Laura
% And it's a special session for games within EvoStar. There's no need to talk about the need to research games, I guess. Also, this probably should go before Motivation - JJ
\subsection{The necessity of Procedural Content Generation}
Computer games are a relatively new form of media whose popularity has been increasing non-stop since they appeared in the 70s. Many challenges have arisen---and will keep doing so---throughout the evolution of the industry, from the early arcades to the most complex modern open-world video games. Developers have come up with all sorts of creative ways to overcome hardware limitations, delivering better graphics and audio. They have pushed the boundaries of the medium by finding new forms of interaction and engaging players with compelling storytelling and original game mechanics. Many of them are related to the fast pace at which the game industry is growing, reaching every passing year to broader audiences that demand a wider range of experiences. Crafting them requires great effort and a high consumption of resources. How do we create a vast amount of content that suits players' expectations with lower investment? If we answer the above question, then we will have increased replayability, adaptative content or reduction of designers' workload. All of which can be tackled using PCG.\cite{togelius2016introduction}

% Please cosider doing a hard-wrap on the text, editing and 
% keeping track of changes in git is going to be easier.

% Consider changing: (others will simply follow each paragraph from now on)
% that suits players expectations with lower investment?
% that meets players' expectations with lower investment?
% DONE - Laura

% The sentence:
% The answer can be replayability, adaptative content or 
% reduction of designers' workload.
% DONE - Laura

% I think what you mean is: IF we answer the above question, then we will have:
% increased replayability, adaptative content and a reduction of designers' workload.
% As it is, this is not easily understood, the above are the result, the benefits not the answer
% DONE - Laura

Replayability, also referred to as replay value, relies on how interesting is playing a game more than once. It is easy to understand why it would be a desirable feature for both players and developers: from the player's point of view, they can extend the game experience further---past the credits roll. For designers, replay value means their product offers more with less manually crafted content.

% how interesting IS playing a game more than once. - Mario 
% DONE - Laura

Games can also engage players by adapting gameplay elements to each individual player. In a literal sense, this would be impracticable. Instead, users are usually presented with options to adjust game elements to their preferred style. What is more interesting, the game itself can change based on in-game player behaviour. It can regulate its difficulty level to fit the player's learning curve or create content that matches the player's taste.

% nstead, users are usually presented with options to adjust to their preferred style.
% nstead, users are usually presented with options to adjust (gameplay, difficulty, scenery) (game elements?) to their preferred style.
% DONE - Laura

Both applications above use PCG as a replacement for human designers. However, PCG can be used as a tool to assist developers. It can suggest what might be a base for later development, enhancing human creativity rather than displacing it. 

%*****************************************
%*****************************************
%*****************************************
%*****************************************
%*****************************************
