\documentclass[runningheads,a4paper]{llncs}

\usepackage[latin1]{inputenc}
\usepackage{graphicx,color,url}
\usepackage[dvips]{epsfig}
\usepackage{verbatim}
\usepackage{tikz}
\usetikzlibrary{shapes,arrows}
\usetikzlibrary{calc,patterns,snakes,decorations.pathmorphing,decorations.markings}
\usetikzlibrary{positioning}

\newcommand{\keywords}[1]{\par\addvspace\baselineskip
\noindent\keywordname\enspace\ignorespaces#1}

\providecommand{\tabularnewline}{\\}

\begin{document}

\mainmatter  % start of an individual contribution

% first the title is needed
\title{}

% a short form should be given in case it is too long for the running head
\titlerunning{}

% the name(s) of the author(s) follow(s) next
%
% NB: Chinese authors should write their first names(s) in front of
% their surnames. This ensures that the names appear correctly in
% the running heads and the author index.
%

\author{A. N. Onymous\inst{1}%
\thanks{No Institute}}
\authorrunning{Anonymous, A}
% (feature abused for this document to repeat the title also on left hand pages)

% the affiliations are given next; don't give your e-mail address
% unless you accept that it will be published
%\institute{Dept. of Computer Architecture and Technology, University of Granada, Spain}
\institute{Anonymous Institute}


%
% NB: a more complex sample for affiliations and the mapping to the
% corresponding authors can be found in the file "llncs.dem"
% (search for the string "\mainmatter" where a contribution starts).
% "llncs.dem" accompanies the document class "llncs.cls".
%

\maketitle

%
%%%%%%%%%%%%%%%%%%%%%%%%%%%%%%%   ABSTRACT   %%%%%%%%%%%%%%%%%%%%%%%%%%%%%%%
%
\begin{abstract}

%\keywords{Videogames, Fuzzy Controller, TORCS, Steering control, Speed computation}
\end{abstract}

%
%%%%%%%%%%%%%%%%%%%%%%%%%%%%%%%   INTRODUCTION   %%%%%%%%%%%%%%%%%%%%%%%%%%%%%%%
%
\section{Introduction}
\label{sec:intro}
 


%The rest of this paper is organized as follows:
%The next section presents the state of the art and related works, and comments their differences with regard to this study, along with its contribution to advance in this research line. TORCS simulator is described in Section \ref{sec:torcs}.
%The proposed fuzzy controller (and its sub-controllers) is detailed in Section \ref{sec:fuzzy_controller}, whereas it is tested in `real' (simulated) races in Section \ref{sec:results}. Obtained results are analyzed and compared 
%with other controllers are given in the same section. Finally, the conclusions and future lines of work are included in Section \ref{sec:conclusions}.


%%%%%%%%%%%%%%%%%%%%%%%%%%%%%%  STATE OF THE ART  %%%%%%%%%%%%%%%%%%%%%%%%%%%%%%
\section{Background and State of the Art}
\label{sec:soa}





%%%%%%%%%%%%%%%%%%%%%%%%%%%%  PROBLEM DESCRIPTION  %%%%%%%%%%%%%%%%%%%%%%%%%%%%
\section{Problem Description: TORCS}
\label{sec:torcs}



%%%%%%%%%%%%%%%%%%%%%%%%%%%%%  CONCLUSIONS  %%%%%%%%%%%%%%%%%%%%%%%%%%%%
%
\section{Conclusions and Future Work} 
\label{sec:conclusions}

\bibliographystyle{splncs03}
\bibliography{angrybirds}



\end{document}
