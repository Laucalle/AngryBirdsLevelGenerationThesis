\documentclass[runningheads,a4paper]{llncs}

\usepackage[latin1]{inputenc}
\usepackage{graphicx,color,url}
\usepackage[dvips]{epsfig}
\usepackage{verbatim}
\usepackage{tikz}
\usetikzlibrary{shapes,arrows}
\usetikzlibrary{calc,patterns,snakes,decorations.pathmorphing,decorations.markings}
\usetikzlibrary{positioning}
\usepackage{amsmath}

\usepackage{tabularx} % better tables
\usepackage{float}
\setlength{\extrarowheight}{3pt} % increase table row height
\newcommand{\tableheadline}[1]{\multicolumn{1}{c}{\spacedlowsmallcaps{#1}}}
\newcommand{\myfloatalign}{\centering} % to be used with each float for alignment
\usepackage{caption}
\captionsetup{font=small} % format=hang,
\usepackage{subfig}
\usepackage[ruled]{algorithm2e}
%\providecommand{\tabularnewline}{\\}
\usepackage{listings}
\usepackage{color}
\usepackage{graphicx}
\newcommand{\keywords}[1]{\par\addvspace\baselineskip
	\noindent\keywordname\enspace\ignorespaces#1}

\definecolor{dkgreen}{rgb}{0,0.6,0}
\definecolor{gray}{rgb}{0.5,0.5,0.5}
\definecolor{mauve}{rgb}{0.58,0,0.82}
\definecolor{gray}{rgb}{0.4,0.4,0.4}
\definecolor{darkblue}{rgb}{0.0,0.0,0.6}
\definecolor{lightblue}{rgb}{0.0,0.0,0.9}
\definecolor{cyan}{rgb}{0.0,0.6,0.6}
\definecolor{darkred}{rgb}{0.6,0.0,0.0}


\lstset{
	basicstyle=\ttfamily\footnotesize,
	columns=fullflexible,
	showstringspaces=false,
	numbers=left,                   % where to put the line-numbers
	numberstyle=\tiny\color{gray},  % the style that is used for the line-numbers
	stepnumber=1,
	numbersep=5pt,                  % how far the line-numbers are from the code
	backgroundcolor=\color{white},      % choose the background color. You must add \usepackage{color}
	showspaces=false,               % show spaces adding particular underscores
	showstringspaces=false,         % underline spaces within strings
	showtabs=false,                 % show tabs within strings adding particular underscores
	frame=none,                   % adds a frame around the code
	rulecolor=\color{black},        % if not set, the frame-color may be changed on line-breaks within not-black text (e.g. commens (green here))
	tabsize=2,                      % sets default tabsize to 2 spaces
	captionpos=b,                   % sets the caption-position to bottom
	breaklines=true,                % sets automatic line breaking
	breakatwhitespace=false,        % sets if automatic breaks should only happen at whitespace
	title=\lstname,                   % show the filename of files included with \lstinputlisting;
	% also try caption instead of title  
	commentstyle=\color{gray}\upshape
}


\lstdefinelanguage{XML}
{
	morestring=[s][\color{mauve}]{"}{"},
	morestring=[s][\color{black}]{>}{<},
	morecomment=[s]{<?}{?>},
	morecomment=[s][\color{dkgreen}]{<!--}{-->},
	stringstyle=\color{black},
	identifierstyle=\color{lightblue},
	keywordstyle=\color{red},
	morekeywords={material, minWidth, maxWidth, width, rotation, type, id, x, y, source, target, version}% list your attributes here
}

\begin{document}

\mainmatter  % start of an individual contribution

% first the title is needed
\title{AngryBirds Level Generation by Means of Evolutionary Search-based PCG}

% Antonio - Improve the title

% a short form should be given in case it is too long for the running head
\titlerunning{Angrybirds levels with EAs}

% the name(s) of the author(s) follow(s) next
%
% NB: Chinese authors should write their first names(s) in front of
% their surnames. This ensures that the names appear correctly in
% the running heads and the author index.
%

\author{A. N. Onymous\inst{1}%
\thanks{No Institute}}
\authorrunning{Anonymous, A}
% (feature abused for this document to repeat the title also on left hand pages)

% the affiliations are given next; don't give your e-mail address
% unless you accept that it will be published
%\institute{Dept. of Computer Architecture and Technology, University of Granada, Spain}
\institute{Anonymous Institute}


%
% NB: a more complex sample for affiliations and the mapping to the
% corresponding authors can be found in the file "llncs.dem"
% (search for the string "\mainmatter" where a contribution starts).
% "llncs.dem" accompanies the document class "llncs.cls".
%

\maketitle

%
%%%%%%%%%%%%%%%%%%%%%%%%%%%%%%%   ABSTRACT   %%%%%%%%%%%%%%%%%%%%%%%%%%%%%%%
%
\begin{abstract}

%\keywords{Videogames, Fuzzy Controller, TORCS, Steering control, Speed computation}
\end{abstract}

%
%%%%%%%%%%%%%%%%%%%%%%%%%%%%%%%   INTRODUCTION   %%%%%%%%%%%%%%%%%%%%%%%%%%%%%%%
%
\section{Introduction}
\label{sec:intro}
%%************************************************

% It's probably the best to eliminate subsections. It saves space, and it's a short paper anyway - JJ

\subsection{What is Procedural Content Generation}

The definition of Procedural Content Generation (PCG) has been broadly discussed but there is no agreement. We have plenty of examples of what \textit{is} and what \textit{is not} PCG, but every definition struggles to cover all cases, either being too inclusive or too exclusive. The one we choose here balances well between the two, defining PCG as \textit{the algorithmical creation of game content with limited or indirect user input} \cite{togelius2011procedural}.

% You can define PCG, NPC, AI the first time they are mentioned

Although PCG often uses Artificial Intelligence (AI) techniques, this definition does not include all uses of it in games. We do not consider Non Playable Character (NPC) behaviour or AI playing agents as content, thus they are not PCG either. Aesthetic elements, game rules, levels, items, stories and characters among others are considered content in this definition.

Note that neither computers nor video games are mentioned in the definition. In fact, PCG has its roots in analogical games. This may conflict with the \textit{limited or indirect user input}, but it is reasonable to assume that following a detailed set of instructions---even if it is done by a human---is not \textit{input}. The underlying concepts used much earlier by non-digital games still prevail in modern video games. Using an algorithm to assemble pre-designed pieces is a common technique in tabletop roleplaying guides---where the algorithm usually consists in several dice rolls---such as \textit{Dungeon \& Dragons}. It is not surprising that one of the early adaptations of PCG to digital platforms aimed to generate monsters and dungeons for physical games.\cite{smith2015analog}

% In a conference paper, we shouldn't probably go down to the level of subsubsection (or, for that matter, subsection) - JJ
% DONE - Laura
% Still 20 pages; we should also eliminate subsections, and leave just top-level sections - JJ 


% This item list should probably be eliminated to leave only the type we are going to use, along with the reference - JJ
There are many PCG methods and it is necessary to look at some traits that characterize and differentiate them from each other: \cite{togelius2016introduction}

\begin{itemize}
	\item \textit{Online/offline}: In online generation, the PCG occurs during the game session, while the user is playing the game. If it is done before the game session or during development, we have offline generation. 
	\item \textit{Necessary/optional}: Procedurally generated content may be part of the essential structure of the game or can be additional to the game experience and can be discarded. In the first one the content is \textit{necessary} and needs to be correct while the latter is \textit{optional}.
	\item \textit{Degree and dimensions of control}: As any other algorithm, a PCG method can have a number of parameters which affect the output. If it uses random numbers, the seed is one of them.
	\item \textit{Generic/adaptative}: Adaptative generation takes into account player behaviour while generic does not. Although there are exceptions, most commercial games choose generic over adaptative.
	\item \textit{Stochastic/deterministic}: Deterministic PCG will produce the same output given the same input, in contrast to stochastic generation which is not as easy to replicate. 
	\item \textit{Constructive/generate-and-test}: Generate-and-test produces potentially correct solutions that are tested and adjusted in each iteration before giving the actual output. Constructive methods build partial solutions and add on them.
	\item \textit{Automatic/mixed authorship}: PCG can be used as an assisting tool for designers, whether the output is used as a base or as an interactive process. Then we talk about mixed authorship, as opposed to the automatic generation where the designer does not take part. 
\end{itemize}

\subsubsection{Search-based Procedural Content Generation}

A special kind of \textit{generate-and-test} approach to PCG is Search-based Procedural Content Generation (SBPCG), which is usually---but not always---tackled with Evolutionary Algorithms. The problems faced by SBPCG are not very far from those encountered in Evolutionary problems. 

The test function does not determine if a solution is valid, but it grades how good the solution is. This is often called \textit{fitness function}. In SBPCG, how to evaluate the quality of a solution has no straightforward answer. It requires formalizing as a \textit{fitness function} how much fun, exciting or engaging certain content is, which are usually based in tricky assumptions.

There are three main classes of \textit{fitness functions} in SBPCG\cite{togelius2010search}:

\begin{itemize}
	\item \textit{Direct Fitness Function} where certain features are extracted from the generated content and directly mapped to a fitness value. Those features must be easily measurable.
	\item \textit{Simulation-Based Fitness Function} where an agent plays through some part of the game that involves the generated content. The fitness is calculated using features from the agent's gameplay. 
	% where an agent plays through some part 
	% DONE - Laura
	
	\item \textit{Interactive Fitness Function} where the fitness value is obtained from the player, whether it be explicitly by asking them or implicitly by measuring certain responses to the game. 
\end{itemize}

% This should go _after_ the definition, and probably before Motivation. - JJ
% DONE - Laura

% And it's a special session for games within EvoStar. There's no need to talk about the need to research games, I guess. Also, this probably should go before Motivation - JJ
% I've shorten the first part of this section, but I don't know if we better just remove it or remove why replayability, adptative contet and assistance tools are interesting for saving resources. Is there need to define PCG in depth in the previous section then? - Laura
% No, really no need. It's a games track, probably people already know what it is. A general definition and focusing on the specific type we are using is more than enough. - JJ

\subsection{The necessity of Procedural Content Generation}
The video game industry is growing at a fast pace, reaching every passing year to broader audiences that demand a wider range of experiences. Developers have pushed the boundaries of the medium by finding new forms of interaction and engaging players with compelling storytelling and original game mechanics. However, crafting those experiences  requires great effort and a high consumption of resources. How do we create a vast amount of content that suits players' expectations with lower investment? If we answer the above question, then we will have increased replayability, adaptative content or reduction of designers' workload. All of which can be tackled using PCG.\cite{togelius2016introduction}

% Please cosider doing a hard-wrap on the text, editing and 
% keeping track of changes in git is going to be easier.
% Sorry, but what do you mean by hard-wrap? - Laura
% It means using editor config to keep line length to a certain number
% of characters, 72 or 80.  Like these two lines - JJ

Replayability, also referred to as replay value, relies on how interesting is playing a game more than once. Players can extend the game experience further---past the credits roll. For designers, replay value means their product offers more with less manually crafted content.

Games can also engage players by adapting gameplay elements to each individual player. In a literal sense, this would be impracticable. The most common way is to present users with options to adjust game elements to their preferred style. What is more interesting, the game itself can change based on in-game player behaviour. It can regulate its difficulty level to fit the player's learning curve or create content that matches the player's taste.

% nstead, users are usually presented with options to adjust to their preferred style.
% nstead, users are usually presented with options to adjust (gameplay, difficulty, scenery) (game elements?) to their preferred style.
% DONE - Laura

Both applications above use PCG as a replacement for human designers. However, PCG can be used as a tool to assist developers. It can suggest what might be a base for later development, enhancing human creativity rather than displacing it. 

%*****************************************
%*****************************************
%*****************************************
%*****************************************
%*****************************************

% Please consider also moving all this file to the main file. It's a
% short paper, and after shortening we are going to have just a few
% paragraphs left - JJ
%%************************************************
\subsection{Motivation}
It is common for AI researchers to turn to games in general as a testing environment for AI techniques. Some of the earliest problems that AI attempted to solve were checkers and chess in the 50s, even before AI was defined and recognized as a field in the Dartmouth workshop in 1956 \cite{nilsson1998artificial}.

What made board games attractive in the first place? They have a rather simple set of rules but winning a game could be a challenging task even for a human brain. It is not surprising that soon video games too drew researchers' attention---along with board games. They offer a wide range of dynamic and competitive elements that resemble real-world problems to some extent.

Of course, this interest works in both ways. Many video games use AI techniques to deliver better experiences, mainly involving NPC behaviour and PCG.

Having this relationship between academia and industry in mind, the IEEE Conference on Computational Intelligence and Games started as a symposium in 2005, and as a conference in 2009. It brings professionals from both fields together to discuss the latest advances in AI and Computational Intelligence and how they apply to games.\cite{ieee-cig}

Among other events, the conference hosts competitions. In 2018, most proposed competitions are centred around playing AI agents for specific games or genres. One of the exceptions is the \textit{3rd Angry Birds Level Generation Competition}. Participants must build computer programs that are able to generate levels for the\textit{ Angry Birds} game.

\textit{Angry Birds} is a mobile game by Finnish company Rovio Entertainment Corporation\cite{angry-birds}, first launched in 2009. The game was a huge success in the \textit{AppStore}---being the most downloaded mobile game in history. Having been ported to several other platforms since, it has an animated movie and way too many \textit{spin offs}---around seventeen. In the game, green pigs have stolen the birds' eggs, and they proceed to rescue them. The pigs have built a variety of defensive structures made out of blocks, so the player has to fire the birds from a slingshot---apparently they never learnt to fly---so the structure is destabilized.

However, what is interesting to us is not its wide success, but how heavily the game relies in gravity to create interesting puzzles. The main challenge is to build stable structures that are robust enough to take more than a single shot before crumbling to the ground.

The ultimate objective of this project is to build a program capable of creating levels for \textit{Angry Birds}. This by itself is too vague to convey the desired approach for the project, so we can break it down to the following, more concise, objectives:

\begin{itemize}
	\item Explore the expressiveness and variability of SBPCG using evolutionary programs.
	\item Adapt the game to extract data from execution, with the aim of evaluating the levels.
	\item Produce stable structures under gravity.
	\item Place other elements on the structures to complete the levels.
\end{itemize} 




% Antonio - I have written a text closer to what a paper needs to include. We must be quite concise and explain just what it is required to justify our work. ;D
% Moreover, for a document as short as a paper is, I would suggest to work on a single TEX file. 

One of the main challenges nowadays in the videogames development is the automatic generation of contents, as games are growing in complexity exponentially \cite{togelius2016introduction}. Thus, there is a vast amount of resources and assets to be created by the artistic staff, designers and programmers, who would profit some tools to automatise or simplify their tasks.

Procedural Content Generation (PCG) brings a possible solution to this situation. It is defined as \textit{the algorithmic creation of game content with limited or indirect user input} \cite{togelius2011procedural}. 
PCG often uses Artificial Intelligence (AI) techniques and normally includes aesthetic elements, game rules, levels, items, stories and character designs among others contents or assets.

The main objective to reach for a videogame is the \textit{player's engagement}, which is mainly based on two factors: replayability (how interesting is playing a game more than once) and adaptability (adapting gameplay elements to each individual player). Replay value strongly depends on the offer of new (and attractive) content for the player, once he/she has finished the main game. The second factor normally lies on the possibility of configuring some game aspects in a menu, such as the game difficulty, character control, or camera options. However, more advanced approaches implement adjustments based on in-game player behaviour: automatically regulating its difficulty level to fit the player's learning curve or creating content that matches the player's taste.
These are also PCG applications.

% Antonio - a paragraph saying which of these factors (replayability or adaptability), or both, are we addressing in this paper.

This paper will present a PCG approach focused on the generation of levels for the videogame \textit{Angry Birds}\cite{angry-birds}, one of the most successful games in mobile phones also ported to many other platforms, following the rules of the international \textit{Angry Birds Level Generation Competition} \footnote{\url{https://project.dke.maastrichtuniversity.nl/cig2018/competitions/#angrybirds}}.

We will consider the so-called Search-based Procedural Content Generation (SBPCG), an approach which produces potentially correct solutions that are tested and adjusted in each iteration before giving the final output. This has been tackled several times in the literature by means of Evolutionary Algorithms [*** Include a couple of REFS ***].

However, the evaluation of the quality of a solution in a SBPCG approach for a game has not a straightforward answer. It requires formalising a \textit{fitness function} involving several factors such as how much fun, exciting or engaging certain content is, which are usually based subjective assumptions.
According to some authors \cite{togelius2010search}, these functions could follow a direct mapping between the content features and its associated value, could be based on a agent's output after it has played or used the generated content, or could be defined as an interactive function in which a human player gives a personal score (or provides a set of values to compute it).

% Antonio - Comment which kind of approach are we following in order to evaluate the generated levels.

Thus, the objectives to address in this paper are as follows:
\begin{itemize}
	\item Explore the expressiveness and variability of SBPCG using evolutionary algorithms for the automatic design of Angry Birds levels.
	\item Adapt the game code to extract data from execution, with the aim of evaluating the created levels.
	\item Produce stable structures under gravity as the core.
	\item Place other elements on the structures to complete the levels.
\end{itemize}

% Antonio - Better define these objectives.

%The rest of this paper is organized as follows:
%The next section presents the state of the art and related works, and comments their differences with regard to this study, along with its contribution to advance in this research line. TORCS simulator is described in Section \ref{sec:torcs}.
%The proposed fuzzy controller (and its sub-controllers) is detailed in Section \ref{sec:fuzzy_controller}, whereas it is tested in `real' (simulated) races in Section \ref{sec:results}. Obtained results are analyzed and compared 
%with other controllers are given in the same section. Finally, the conclusions and future lines of work are included in Section \ref{sec:conclusions}.


%%%%%%%%%%%%%%%%%%%%%%%%%%%%%%  STATE OF THE ART  %%%%%%%%%%%%%%%%%%%%%%%%%%%%%%
\section{Background and State of the Art}
\label{sec:soa}
%************************************************
Let's have a look at some participants in previous editions of the Angry Birds Level Generation Competition.

\subsection{Constructive approach}
In the two solutions proposed by Matthew Stephenson and Jochen Renz \cite{stephenson2017generating} \cite{stephenson2016procedural} the structures displayed on the level are constructed from the top down, in several phases. 

First, they build a structure recursively, each row composed of a single type of block (with a fixed rotation). The likelihood of selecting a certain block is given by a probability table. Then, the blocks are placed using a tree structure, where the first selected block is the peak and blocks underneath it are split into subsets that support the previous row with one, two or three blocks. This ensures local stability, but not global stability, which is tested once the whole structure is completed. The second solution also tries to add variety by replacing some blocks with others of the same height.

After that, other objects (pigs and TNT) are placed. Potential positions for those objects are recorded, first trying to place them in the centre of each block, then right above the edges of it, checking if there is enough space. Available positions are ranked based on structural protection, dispersion, etc. and then filled with the desired number of objects.

The second solution also selects material---which can be stone, ice or wood---based on trajectory analysis, clustering, row grouping, structure grouping or randomly. Weak points are set to stone material and the rest using one of the previous strategies. Trajectory analysis-based strategy sets to the same material all blocks in the trajectory of a shot aimed at a particular pig. Clustering strategy takes a random block, sets its material and propagates to the surrounding blocks that have not been assigned yet. Row grouping and structure blocking apply the same material to a whole row or structure respectively.

Materials also determine which kind of birds will be used in the level, since some birds are more efficient against certain materials than others. The number of birds available for the player to solve the level is estimated using AI agents designed for a different competition. If the AI agents are unable to solve the level, it is discarded. The lowest number of birds used by the agents is the chosen for the level.

The probability table was tuned using search-based optimization methods.

\subsection{Search-based approach}

Lucas Ferreira and Claudio Toledo\cite{ferreira2014search} present a solution based on SBPCG using a genetic algorithm and a game clone developed in Unity Engine to evaluate the levels.

In the genetic algorithm individuals correspond to levels, each represented by an array of columns. Each column is a sequence of blocks, pigs and predefined compound blocks, using an identification integer. This representation also includes the distances between different columns.

The population is initialized randomly following a probability table which defines the likelihood of a certain element being placed in a certain position inside a column.

For the evaluation, levels are executed in the game which stores data about the simulation. The fitness function is described as:

$$ f_{ind} = \frac{1}{3}(\frac{1}{n} \sum_{n-1}^{i=0}{v_i}+\frac{\sqrt{(|b|-B)^2}}{Max_b-B}+ \frac{1}{1-|p|})$$ 

where $v_i$ is the average magnitude of the velocity vector for block $i$, $|b|$ is the total of blocks in the level and $|p|$ the count of pigs. The rest are parameters: $B$ the desired amount of blocks and $Max_b$ the maximum number of elements.

The recombination process uses an Uniform Crossover where the new individual is generated picking for each position a column from one of the parents which occupies the same position. When the parents are not the same length, the remaining columns are selected with a $50\%$ of probability. Mutation simply changes with a certain probability each element of the individual randomly (either being a column element or the distance between columns).

\subsubsection{Open Source Simulator}

The fitness evaluation requires a simulation to test the behaviour of the levels under the game's physics. Angry Birds is not open source software and so the code is not available; therefore, a game clone was developed to fill this gap, using the Unity Engine.

Levels are described in XML files which the game takes as an input. They are parsed to run the simulation which then generates new XML files as an output. Those files contain information about the execution of a certain level such as average velocity of each element, the amount of collisions and the final rotation. 
%*****************************************
%*****************************************
%*****************************************
%*****************************************
%*****************************************



% If possible, add more entries about evolution of free-form structures, which was after all the motivation for this particular algorithm - JJ
% In general or from the competition? - Laura

%%%%%%%%%%%%%%%%%%%%%%%%%%%%  PROBLEM DESCRIPTION  %%%%%%%%%%%%%%%%%%%%%%%%%%%%
\section{Problem Description}
\label{sec:aibirds}

\input{Modifications}
%************************************************
\subsection{Evolutionary Algorithm}\label{ch:Representation}

The most challenging part of the level generation process is to create stable structures. As obvious as it is, the main feature of a stable level is that it is not in motion. So it seems reasonable to evaluate their stillness as opposed to their speed.

$$fitness_{ind} = \frac{1}{|V|}\sum_{i=0}^{b}{V_i} + P_{broken}\cdot(b-|V|)$$

As mentioned in section \ref{ch:gameApdaptation} the game output provides the average magnitude of velocity for each block. This set is noted as $V$, with $|V|$ being the cardinality of the set. The number of blocks in an individual is $b$ and it can differ from the cardinality of $V$ since broken blocks are not tracked. The number of broken blocks is $b-|V|$ and it is multiplied by a penalization factor, since a level whose blocks break without user interaction would not be considered valid. This happens when a block free-falls from a certain height or collides with a falling object. $P_{broken}$ is set to $100$ since objects in a level do not usually reach that velocity, therefore it will separate non-valid levels from potentially good ones.

% When we say length of a vector, we think about the norm (the actual lenght) 
% for instance the hypotenuse in a 2D vector. I think here we are talking about the
% number of elements. If V is a set, then the cardinality |V| is right. 
% From your "calculateFitness" this is len so it is a list. 
% I was confused because we are talking about magnitude, velocity, etc. 
% Consider changing the name of V to an ordered set? - Mario
% Done - Laura
  

Not all levels are evaluated using a simulation, since it is a costly process. Although we may know that a given level will perform poorly in the simulation, removing it from the population will cause a high loss of diversity. For that matter, before testing a level, there are some indicators that a level would not be suitable and can be skipped in the simulation. Those are its distance to the ground and the number of blocks that overlap.

If the lowest object is not close to the ground is very likely that all blocks will break in the impact. Levels that have all their blocks higher than a certain threshold will not be simulated in the game. The threshold used is $0.1$ in game units and the penalty applied to the fitness value is $10$:
$$f_{distance} = 
\begin{cases}
	P_{distance}\cdot D_{lowest}, & \text{if } D_{lowest} > threshold\\
	0, & \text{otherwise}
\end{cases}
 $$

% This is question about your code. In there each block individual
% has a penaly attribute. This is a penalty value or is something that is
% calculated? This value is the same for all individuals? If it is maybe
% it belongs on the experiment or evolution object.
% There is a penalty that is equal for all of them. It was that way to make the fitness value calculation of each individual independent. I am pretty sure it would be easy to change. Does it need to be change before the paper submission date? 


The other measure is the number of overlapping blocks. The separating axis theorem\cite{ericson2004real} determines if two convex shapes intersect. It is commonly used in game development for detecting collisions. A level with blocks that occupy the same space is not likely to be stable, as the Unity Engine will solve the issue moving the blocks until there is no collision. Since Unity Engine is not open source and there is no documentation on how exactly those collisions are solved, we assume that a precise prediction of the positions of the blocks is not possible and therefore the fitness value obtained could be inaccurate. A penalization is applied and the level is not simulated in-game. In this case it is $f_{overlapping} = P_{overlapping} \cdot N_{overlapping}$ where the first factor is a penalty set to $10$ and the second is the number of blocks that overlap with each other. 

If both $f_{distance}$ and $f_{overlapping}$ are $0$ then the level is suitable for the simulation and fitness is calculated as $fitness_{ind}$. This would be considered \textit{overpenalization} but exploring infeasible regions entails a serious overhead that we need to minimize. On the other hand, levels with multiple blocks broken during the simulation are not feasible either but running the simulation is necessary. In this case, the penalization does not prevent the region to be explored.

Previous approaches to this problem (studied in section \ref{sec:soa}) provide fairly constrained outputs. The constructive method presented creates pyramid-like structures and, even though there is a variety of levels, the method is highly specialized in this kind of theme. The search-based approach only produces tower structures adding some variety by having pre-built compound blocks.

Since one of the objectives of this project is to explore the expressiveness and variability of SBPCG, it seems reasonable to use a flexible representation. We will try a less directed search than previous solutions while keeping a simple representation. 

Individuals are composed by a list of blocks, each of them being a gene. Special pieces such as platforms, TNT boxes or pigs are not considered in this phase. The building blocks for the game have several attributes that characterize them: 

\begin{itemize}
	\item Type: there are eight regular blocks that can be placed in the level with distinct shapes or sizes. Represented as an integer between $0$ and $7$.
	\item Position: coordinates $x$ and $y$ of the centre of the block in game units.
	\item Rotation: rotation of the block in degrees. Here four different rotations are considered, 0º, 45º, 90º or 135º represented as integers between $0$ and $3$.
\end{itemize}

Using this representation a gene will be formed by two integers and two floating point numbers.
The position of the corners of the block is frequently required, so it is stored along with those attributes even though it can be calculated using the size, the position and rotation of the block. 

There are three types of materials in the game, which determine the durability of the block. However, this does not affect their stability, so it will remain constant for now as \textit{wood} material.

Individuals are a collection of genes, in the same way a level is a collection of building blocks. The number of blocks is variable and the order in which they are listed is not important. 

As previously stated, only promising individuals are tested in-game, meanwhile those who are not, are penalized instead. This penalty is stored, separately from the fitness value for the individual. The reason for this is that it may change over generations. The goal of the penalization is to maintain fitness value of not tested level above ---it is a minimization problem--- the in-game tested levels, so the starting point for fitness of such individuals is the worst in-game score.

This penalization is calculated using the distance of the lowest block to the ground, which can be easily obtained, and the number of blocks that collide. This requires a bit more of computation, so it will be stored and set in the initialization of the individual. When a gene is modified, the number of overlapping blocks is recalculated for that specific change.

Considering all of the above, the chromosome object is composed by:
\begin{itemize}
	\item A list of genes.
	\item A fitness value.
	\item A penalty (set to $-1$ for in-game evaluated levels).
	\item Number of overlapping blocks (calculated).
\end{itemize}

\subsection{Genetic Operators}

Initialization is done randomly, with each individual having a random number of genes. Those genes can be initialized using several methods. 

\begin{itemize}
	\item Random: selects a random number for each attribute of the gene.
	\item No Overlapping: also selects a random number but the gene is only added to the chromosome if it does not overlap with an already existing gene.
	\item Discrete: selects a random number for type and rotation, but the position must be multiple of the dimensions of the smallest block (blocks will be aligned).
	\item Discrete without overlapping: it combines the second and third initialization method.
\end{itemize}

Parents are selected using tournaments. Two individuals are chosen from the population and the best of them will be a parent in this generation. This is repeated until a certain percentage of pairs have been reached. It is important to note that individuals chosen are not removed from the population and therefore they can appear several times in the list of parents. 
% Is there a reason for this? For small populations some individual could
% be the parent of every member of the next generation :)  - Mario
% The reason was we were getting very poor results, beacuse we start with very few fairly good individuals. Should this be mentioned here? - Laura
% Yes, you should mention it. Beacuse this a good problem for future work. 
% For instance you could try to do this only at the begining and after several
% generations change it to something not as extreme. This means that the parameters
% of the algorithm are dynamic or adapted. - Mario
% OK, I added it to the Future Work section. 

Once the parents have been selected, we implement two different methods of combination.
\begin{itemize}
	\item Sample Crossover: gives a single individual per parent pair. It takes all genes from both parents---excluding genes that are repeated---and randomly takes a number of them to create the new individual. The number of blocks is the minimum between the maximum number of blocks allowed, the mean of the two parent individuals and the number of distinct genes.
	\item Common Blocks: produces two individuals. The common genes to both parents are passed on to both children. The remaining genes are randomly distributed to each child, half to one and half to the other. \label{ga:cross2}
\end{itemize}

There are four different mutations:

\begin{itemize}
	\item Type: type is represented as an integer, so it adds or subtracts one to the current value.
	\item Rotation: similarly to type mutation.
	\item Position X: a real value between $0$ and $1$---excluding $0$---is added or subtracted from the value of the position X.
	\item Position Y: same as position X mutation, for position Y.
\end{itemize}

The new generation is selected using an elitist strategy. Best individuals in both the old population and their offspring pass on the next generation, maintaining the size of the population.


%************************************************

%*****************************************
%*****************************************
%*****************************************
%*****************************************
%*****************************************

\input{Framework}
%************************************************
\section{Experimentation and Results}\label{ch:res}
Each experiment was based on an hypothesis formulated after the observation of previous results. The structure of this section reflects this methodology.

Four experiments have been carried out, with more than 15 executions per experiment. Table \ref{t:resOver} shows an overview of the results.
\begin{table}[H]
	\myfloatalign
	\begin{tabular}{ccccc}
		& \textbf{E1} & \textbf{E2} & \textbf{E3} &\textbf{E4} \\ \hline
		\textbf{Time (h)} & 0.89 & 1.002 & 1.76 & 5.03 \\  \hline
		\textbf{G} &  100.0 & 155.087 & 76.625 & 365.929 \\  \hline
		\textbf{Best} & 61.334 & 110.66 & 0.0015 & 0.0018 \\  \hline
		\textbf{Avg} & 383.701 & 327.547 & 0.54 &  0.203 \\  \hline
		\textbf{Worst}  & 510.515 & 367.895 & 0.828 & 0.2997\\  \hline

		 \hline
	\end{tabular}
	\caption{Summary of results, G: number of generations}
	\label{t:resOver}
\end{table}

\subsection{First Experiment}
The premise of the first experiment is that our basic EA should be able to minimize the movement of the blocks placed on the level and the flexibility of the representation should allow variety in the structures. This may be optimistic, but we need an estimation as a starting point, and this experiment will serve the purpose. 

The EA in this experiments uses:
\begin{itemize}
	\item  initialization with the discrete method described earlier.
	\item  basic sample crossover.
	\item  all four mutations.
	\item  elitist replacement.
	\item  selection using tournaments.
\end{itemize}

The parameters are:
\begin{itemize}
	\item Population size: 100
	\item Number of generations: 100
	\item Percentage of parents: 0.5
	\item Percentage of type mutations: 0.5
	\item Percentage of rotation mutations: 0.5
	\item Percentage of axis x mutations: 0.5
	\item Percentage of axis y mutations: 1
\end{itemize}

The results suggest that the hypothesis was not correct. The average best solution has a fitness value of 61.334 (as shown in Table \ref{t:resOver}) which indicates that probably most levels have blocks falling (and even breaking) when loaded. The standard deviation of this measure is 133.0209 which implies that while some executions performed poorly, some others may be good. Even the best levels have blocks that break after loading so they would not be valid. However, we can tell that there is variety in the structures, since they clearly differ from each other. 

In the experiment the only termination condition was reaching the maximum number of generations. However, looking at the results, it seems that the execution ended before the population stabilised or converged. Since mutation percentages are high it is normal that convergence where every single individual is the same one, is not reached. However, it could be possible that the population is stable, where every child has a greater fitness value than its parents, therefore no new members are allowed. If there are no new individuals and the population is completely \textit{stuck}, the fitness value of the worst individual should be the same over several generations. In Figure \ref{f:grahp1} we can see that is not the case in average. Although some populations do remain the same for several generations close to the maximum, most of them do not.

\begin{figure}[H]
	\centering
	\includegraphics[scale=0.55]{exp1_worstIndv.png}
	\caption{In grey, different executions of the first experiment 1}\label{f:grahp1}
\end{figure}

\subsection{Second Experiment}

Most executions from the previous experiment reached the maximum number of generations without stabilizing or converging. This means the EA may need a larger number of generations to fully evolve a solution. This is the hypothesis for the second experiment.


The set of operators is the same as the previous one. The parameters remain unchanged except for:
\begin{itemize}
	\item Number of generations: 1000
\end{itemize}
We also added two stop conditions: 10 generations without changes (stable population) or best fitness value below 0.01.


Although best levels obtained with this experiment are better than those evolved in the first experiment, the bad solutions have a really high fitness value. In table \ref{t:resOver}, we can see that average fitness of levels produced with this version of the EA is worse than the ones generated in the first experiment. It suggest that populations can be stuck for many generations before making any type of improvement. Any of the generated levels have a fitness below 0.01 or reached maximum number of generations, which means the termination criteria that stopped the evolution was that the population was stable, without any new individuals added for 10 generations. 
\begin{figure}[H]
	\centering
	\includegraphics[scale=0.5]{exp2_explication.png}
	\caption{Best individual evolution for all executions, grouped by number of generations}\label{f:grahp2}
\end{figure}
Figure \ref{f:grahp2} represents the evolution of the best individual of each execution. Most of them have no more than 50 generations, therefore the hypothesis for this experiment is not correct. Short evolutions show that the best individual at initialization is very similar to the last one. Slight improvements may be achieved by small mutations but it seems difficult for new generations to outperform previous ones. We can appreciate that significant improvements are most common in those executions with a poor initial population. Even the ones with several hundreds of generations struggle to improve the initial population.

\subsection{Third Experiment}
The previous experiment proved that the problem is not with the number of generations. Then it is more likely that there this EA is biased towards exploration rather than exploitation. The genetic operators are failing to create new individuals that inherit good traits from their parents. A new crossover operator could shift the focus to exploitation.


For the third experiment, the change introduced is in the crossover operator used. It was described in section \ref{ga:cross2}. The rest of the operators as well as the termination criteria are kept the same. 


Table \ref{t:resOver} shows that the results have radically improved as the average fitness of the best solutions drops to $0.0015$, a decrease of almost 100\%. Additionally, it took less generations in general to reach those results. However, executions took longer on average, which makes sense given that a greater number of individuals would have been simulated. The average fitness of the population and the worst individual have similar values now, which suggest that in most executions the population did converge.

The levels are stable and the blocks do not fall when loading the level, but they could arguably be considered structures, since most of them consist in a few blocks displayed on the floor. The average amount of blocks is $6.26$, which is really close to the minimum amount of blocks allowed. However, given the proposed fitness function, it is completely logical that the evolution leads to this kind of arrangements. The more objects placed on the level, the more likely the individual is to not meet the requirements imposed by the constraints. It also makes sense to place objects near the ground, instead of one on top of the other.

\subsection{Fourth Experiment}
The previous EA generates levels with a number of blocks that tends to the minimum of blocks allowed which is 5. Such a small number of elements does not create interesting structures. A higher number of blocks may lead to more appealing levels. We test this in the fourth experiment.


The only parameter that was changed for this experiment is the minimum number of blocks from 5 to 10. 


The first thing to notice in this experiments (Table \ref{t:resOver}) is the increase of the average execution time: it went from 1.76 hours in the third experiment to 5.03 hours in this one. The time spent running the simulation for each population in this experiments and in the previous one should be similar. However, the number of executions drastically increased too. There is no doubt that placing at least ten objects is more difficult than placing 5. The average best fitness value slightly rose, while the average and worst values are lower. This suggest that the latest generations of this EA are less diverse than those from the third experiment.

%************************************************

%*****************************************
%*****************************************
%*****************************************
%*****************************************
%*****************************************




%%%%%%%%%%%%%%%%%%%%%%%%%%%%%  CONCLUSIONS  %%%%%%%%%%%%%%%%%%%%%%%%%%%%
%
\section{Conclusions and Future Work} 
\label{sec:conclusions}
%************************************************
\subsection{Conclusion}

Here, we briefly recap several aspects of the development of this project. First, we reviewed the context of the problem: PCG. In section \ref{sec:intro}, we looked at the main reasons for its existence: from replacing human creativity where it cannot reach (unlimited or personalized) to enhancing and speeding the creative process. In some cases, PCG can  be considered a kind of creative expression for itself. The technical background to approach the problem is discussed in section \ref{sec:soa}.


This project was developed with a set of objectives in mind (stated in section \ref{motivation}). The first goal was to explore the expressiveness and variability of SBPCG with evolutionary techniques. Perhaps the level of achievement of this one is not as easily measurable as the other three. It could seem like this objective has not been fully accomplished: only one SBPCG method has been implemented and tested. However, it was not in the scope of the project to test SBPCG in general, but in this particular case, for this particular game. Considering this, the method studied was sufficiently general and flexible to draw some conclusions about the topic. SBPCG methods are a potential good solution to offline content generation but it requires a great amount of problem-specific knowledge. Like any other form of creative work, the biggest issue may be how to measure how good, creative or enjoyable is the piece. The more rules the author adds, expressiveness starts to get lost as the results are variations of the same idea. However, it is crystal clear, from the experiments run in this project, that a lack of knowledge will lead to unexpected and even disappointing outcomes.

The second objective, adapting the game to extract data from execution, was certainly achieved. It was also a basic requirement to proceed with the rest of them. The game does provide the data, as long as the input is correctly structured. Although the original intention was to make it available on Linux, currently it can only be executed with the desired behaviour on Windows. Another issue is that the simulation is not easily adaptable. If the fitness function of the EA is changed and needs data that is not included in the output right now, the game would have to be changed and compiled again.

Producing stable structures under gravity was the third objective, and can be considered half a victory. The levels generated in the last experiments were undeniably stable. Whether or not they can be considered structures, that is arguable. Again, the main issue is how we evaluate the levels. In fact this is a matter of how we define what an Angry Birds' level \textit{is} and if that definition matches the fitness function. A lot of elements were correct, but the definition was not complete.

This idea brings us to the last objective, that was not achieved. Without results that match the definition of an Angry Birds level, placing the remaining objects would not have made the level playable. The partial achievement of the third goal, blocked the fourth one as this objective was dependent on the third one.

To conclude on an optimistic tone, this work provides an interesting insight into the SBPCG, through the completion---and failure---of those four goals.

\subsection{Future work}

In order to improve the results of the method, different constraints could be expressed as multiple objectives. Overlapping blocks, velocity and height could be treated as minimization objectives.

If we pay attention at the stages of evolution in this project, there is also room for improvement in the genetic operators. For example, the initialization produces a small amount of valid individuals which suggested that an elitist strategy for selection would work best. However, new experiments will help to better balance exploration and exploitation. 

Once we had a generator that meets our expectations, performance could be enhanced by the analysis of the best set of initial parameters using Analysis of Variance (or ANOVA), as presented in \cite{estevez2017statistical}. 

Some options were discarded for time limitations, so these could be main areas of development. The chromosome representation using generative grammatical encoding\cite{hornby2001advantages}, although it would radically change the structure of the method, might ensure that generated levels are consistent. This would require carefully selection the operators that would be the building blocks of the generative grammar.

Another important issue that needs to be addressed is the time performance. Right now the simulation is the main bottleneck in the execution. One way to speed up the process can be \textit{cleaning} the current simulation, getting rid of any unnecessary asset while maintaining the bare minimum. However, this will not solve the problem, since the content generator will still need to launch an external executable.

The communication between the simulation and the content generator is through read and write operations on disk, instead of memory. This could be avoided if both tools were integrated in one. If we aim for an online automatic generator, the generator should be integrated in the game. However, if our goal is to generate levels for mixed authorship, as an assistance to developers, it may be a better idea to integrate the simulation in the generator. This could be done by approximating in-game physics with real physics, as described in \cite{blum1970stability}. 


%************************************************

%*****************************************
%*****************************************
%*****************************************
%*****************************************
%*****************************************


\bibliographystyle{splncs03}
\bibliography{angrybirds}



\end{document}
