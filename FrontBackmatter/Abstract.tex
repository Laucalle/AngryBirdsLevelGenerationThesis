%*******************************************************
% Abstract
%*******************************************************
%\renewcommand{\abstractname}{Abstract}
\pdfbookmark[1]{Abstract}{Abstract}
\begingroup
\let\clearpage\relax
\let\cleardoublepage\relax
\let\cleardoublepage\relax

\chapter*{Abstract}
The goal of this project is to build stable structures under gravity for the physics based puzzle game Angry Birds. A search-based procedural level generation method was developed using evolutionary algorithms. Different representations and operators were considered and studied. To evaluate the stability of the levels, they are executed in an adaptation of an open source version of the game (available in \cite{sciencebirds-adapt}). An evolutionary framework was implemented to fit the requirements of the problem (available in \cite{ab-level}). To test the method, four experiments were carried out, obtaining a variety of stable structures.


\paragraph{Keywords:} Search-Based Procedural Content Generator, Evolutionary algorithm, Game development, Angry Birds, Level generation

\vfill
\begin{otherlanguage}{ngerman}
\pdfbookmark[1]{Resumen}{Resumen}
\chapter*{Resumen}
El objetivo de este trabajo es construir estructuras estables para el juego de puzzles basado en físicas Angry Birds. Para ello, se desarolla un método de generación procedural de niveles basado en exploración de espacios de búsqueda utilizando algoritmos evolutivos. La estabilidad de los niveles se evalúa en una simulación dentro del motor de juego, utilizando una versión \textit{open source} del juego original (disponible en \cite{sciencebirds-adapt}). Se ha implementado un marco de trabajo para algoritmos evolutivos que se ajusta a las exigencias del problema (disponible en \cite{ab-level}). Para probar el método se han realizado cuatro experimentos, obteniendo una variedad de estructuras estables. 

\paragraph{Palabras clave:} Generación procedural de contenido basado en búsqueda, Algoritmos evolutivos, Desarrollo de videojuegos, Angry Birds, Generación de niveles
\end{otherlanguage}

\endgroup

\vfill
