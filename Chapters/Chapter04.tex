%************************************************
\chapter{Open source game adaptation}\label{ch:gameApdaptation}
As a part of their solution, Lucas Ferreira and Claudio Toledo\cite{ferreira2014search} developed an open source simulation for the \textit{Angry Birds} game. Since that implementation is available on Github, it will be used as a starting point for this part of project.

\section{Pre-existing game overview}
The game is implemented using the Unity Engine and its code is mainly written in C\#. Levels are not part of the game itself and instead they are parsed from xml files contained in an specific folder in the game hierarchy which corresponds with \textit{Assets/StreamingAssets/Levels} directory. Each level is described in a different file, similar to listing \ref{xmlsample}
\lstset{language=XML}
\begin{lstlisting}[caption=Sample level input to show format, label=xmlsample]
<?xml version="1.0" encoding="utf-16"?>
<Level width="2">
	<Camera x="0" y="0" minWidth="20" maxWidth="25">
	<Birds>
		<Bird type="BirdRed"/>
		<Bird type="BirdBlack"/>
	</Birds>
	<Slingshot x="-7" y="-2.5">
	<GameObjects>
		<Block type="RectMedium" material="wood" x="2.78" y="-3.18" rotation="0"/>
		<Block type="RectFat" material="wood" x="2.75" y="-2.66" rotation="90.00001"/>
		<Block type="SquareHole" material="stone" x="2.04" y="-2.3" rotation="0"/>
		<Block type="RectSmall" material="wood" x="4.05" y="-3.08" rotation="0"/>
		<Pig type="BasicSmall" material="" x="2.74" y="-1.98" rotation="0"/>
	</GameObjects>
</Level>
\end{lstlisting}
The xml file sets a number of parameters for the level, some of them are key to level generation and others that will be ignored by the generator---and will be fixed, as it will be stated in following chapters--- but still needed for the simulation:
\begin{itemize}
	\item Level width 
	\item Camera position
	\item Birds list, each one with its type (for our objective of making stable structures this will remain constant, although it is important from a game designer's point of view)
	\item Slingshot position
	\item Game objects list, this one is the most important for level generation. It can contain blocks, pigs or TNT.
\end{itemize}

Those elements are then used to build a level, placing the objects in the specified coordinates (note that 0,0 correspond to the centre of the screen, being the ground at $y=-3.5$in game world units).

Although the simulation described in \cite{ferreira2014search} does offer an output and it is automated to run through all levels, the implementation provided does not. However, some of these features can be found in the source code.

\section{Changes in the adaptation}
Considering the functionality of the existing code, there are a few issues preventing us to use the game as a simulation:

\begin{itemize}
	\item (Issue \#1) Human interaction is needed to proceed: there is a main title screen and a level selection.
	\item (Issue \#2) A level will only be finished if the user skips it or there is no pigs left. If there were not pigs in the level description, the game may have inconsistent behaviour, displaying the \textit{Level cleared} banner or crashing.
	\item (Issue \#3) Once a level has been skipped or cleared, data from execution is not stored.
	\item (Issue \#4) Skipping the last level results in loading the first level again. 
\end{itemize}


\subsection{Output xml}

Modifying the simulation to write data output---Issue \#3--- is fairly simple since the functions were already in the code. However, the data it was being stored was not enough for its purpose. Output xml level was a valid input xml level, which misses some key information about the execution and in this case there were not much interest in obtaining a new valid level. The desired output would look like listing \ref{xmlsampleOUT} (note that encoding was also changed from original):

\lstset{language=XML}
\begin{lstlisting}[caption=Sample level output to show format, label=xmlsampleOUT]
<?xml version="1.0" encoding="utf-8"?>
<Level width="2">
	<Camera x="0" y="0" minWidth="20" maxWidth="25" />
	<Birds>
		<Bird type="BirdRed" />
	</Birds>
	<Slingshot x="-5" y="-2.5" />
	<GameObjects>
		<Block type="RectSmall" material="wood" x="3.78" y="-3.294594" rotation="90" id="0" aVelocity="6.974828E-08" />
		<Block type="SquareTiny" material="wood" x="3.78" y="-2.551131" rotation="90" id="1" aVelocity="1.464643E-07" />
	</GameObjects>
</Level>
\end{lstlisting}

To calculate average magnitude of velocity for each block, it should be measured at regular time intervals; but, since this and other processes seem to be frame rate dependant, this was achieved by forcing the game to a constant frame rate. Blocks that were destroyed on execution are not present in the output.

Encoding was changed from \texttt{utf-16} (C\# xml reader/writer default) to \texttt{utf-8} (only format available for Python's Standard Library xml reader).

\subsection{Automatization}

The next logic step is getting rid of interaction. The simulation we are interested in does not require a player (human or AI agent) to play the level since the goal is to know how the level behaves under the game's physics that do not need ---and they do not--- to match real physics. 

A level will run for a certain amount of time, and then continue to the next one, skipping the main menu and level selection. When the simulation reaches the last level it will end the execution.

Unity Engine projects are usually divided into scenes. This one has a different scene for the main menu, level selection, loading screen, levels and failure screen. Removing scenes from the screen flow causes side effects such as the first level crashing, so instead the same functions invoked by the \acs{GUI} are called after initialization takes places in both scenes (main menu and level selection) before the first level. This solves interaction with menus (Issue \#1). 

In order to skip levels, and keeping in mind the velocity if the blocks is already being tracked, the game will load the following scene ---the next level--- once all blocks reach velocity $0$, which means the blocks have stabilized. Just in case this condition is never met, there is also a time limit of $10s$ for each level (Issue \#2). 

The remaining issue (Issue \#4) is solved just by checking how many levels have been loaded and the index of the currently playing level.

%************************************************

%*****************************************
%*****************************************
%*****************************************
%*****************************************
%*****************************************
