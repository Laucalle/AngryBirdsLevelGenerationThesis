%************************************************
\chapter{Introduction}\label{ch:introduction}
%************************************************

\section{The necessity of Procedural Content Generation}
Computer games are a relatively new form of media which popularity has been increasing non-stop since they appeared in the 70s. From the early arcades to the most complex modern open-world video games, the industry has encountered many challenges and as it evolves more arise. Developers have come up with all sort of creative ways to overcome hardware limitations, delivering better graphics and audio. They have pushed the boundaries of the medium finding new forms of interaction, engaging players with compelling storytelling and original game mechanics. Many of them are related to the fast pace at which the game industry is growing, reaching each year to a broader audience that demands a wide range of experiences. Crafting those requires great effort and high consumption of resources. How do we create a vast amount of content that suits players expectations at lower investment? The answer can be replayability, adaptative content or reduction of designers' workload. All of which can be tackled using \acf{PCG}.

Replayability, also referred to as replay value, lies in how interesting is playing a game more than once. It is easy to understand why it would be a desirable feature for both players and developers. From the player's point of view, they can extend the game experience further past the credits roll. For designers, replay value means their product offers more with less manually crafted content.

Games can also engage players by adapting gameplay elements to each individual player.  In a literal sense, this would be impracticable. Instead,  users are presented with options to adjust to their preferred style in some cases, in others it is the game itself which changes based on in-game player behaviour.

Both applications above use \ac{PCG} as a replacement for human designers. However, \ac{PCG} can be used as a tool to assist developers, suggesting what might be a base for later development.

\section{What is Procedural content Generation}
\section{Evolutive Algorithms}
%*****************************************
%*****************************************
%*****************************************
%*****************************************
%*****************************************
