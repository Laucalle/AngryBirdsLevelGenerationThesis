\documentclass[sigconf]{acmart}

\usepackage{booktabs} % For formal tables

\usepackage[latin1]{inputenc}
\usepackage[T1]{fontenc}
\usepackage{libertine}
\usepackage{graphicx,color,url}
\usepackage[dvips]{epsfig}
\usepackage{verbatim}
\usepackage{tikz}
\usetikzlibrary{shapes,arrows}
\usetikzlibrary{calc,patterns,snakes,decorations.pathmorphing,decorations.markings}
\usetikzlibrary{positioning}
\usepackage{amsmath}
\usepackage{tabularx} % better tables
\usepackage{float}
\setlength{\extrarowheight}{3pt} % increase table row height
\newcommand{\tableheadline}[1]{\multicolumn{1}{c}{\spacedlowsmallcaps{#1}}}
\newcommand{\myfloatalign}{\centering} % to be used with each float for 
%alignment
\usepackage{caption}
\captionsetup{font=small} % format=hang,
\usepackage{subfig}
\usepackage[ruled]{algorithm2e}
%\providecommand{\tabularnewline}{\\}
\usepackage{listings}
\usepackage{color}
\usepackage{graphicx}

%\newcommand{\keywords}[1]{\par\addvspace\baselineskip
%	\noindent\keywordname\enspace\ignorespaces#1}

\definecolor{dkgreen}{rgb}{0,0.6,0}
\definecolor{gray}{rgb}{0.5,0.5,0.5}
\definecolor{mauve}{rgb}{0.58,0,0.82}
\definecolor{gray}{rgb}{0.4,0.4,0.4}
\definecolor{darkblue}{rgb}{0.0,0.0,0.6}
\definecolor{lightblue}{rgb}{0.0,0.0,0.9}
\definecolor{cyan}{rgb}{0.0,0.6,0.6}
\definecolor{darkred}{rgb}{0.6,0.0,0.0}

\lstset{
	basicstyle=\ttfamily\footnotesize,
	columns=fullflexible,
	showstringspaces=false,
	numbers=left,                   % where to put the line-numbers
	numberstyle=\tiny\color{gray},  % the style that is used for the 
	%line-numbers
	stepnumber=1,
	numbersep=5pt,                  % how far the line-numbers are from the code
	backgroundcolor=\color{white},      % choose the background color. You must 
	%add \usepackage{color}
	showspaces=false,               % show spaces adding particular underscores
	showstringspaces=false,         % underline spaces within strings
	showtabs=false,                 % show tabs within strings adding 
	%particular underscores
	frame=none,                   % adds a frame around the code
	rulecolor=\color{black},        % if not set, the frame-color may be 
	%changed on line-breaks within not-black text (e.g. commens (green here))
	tabsize=2,                      % sets default tabsize to 2 spaces
	captionpos=b,                   % sets the caption-position to bottom
	breaklines=true,                % sets automatic line breaking
	breakatwhitespace=false,        % sets if automatic breaks should only 
	%happen at whitespace
	title=\lstname,                   % show the filename of files included 
	%with \lstinputlisting;
	% also try caption instead of title  
	commentstyle=\color{gray}\upshape
}


\lstdefinelanguage{XML}
{
	morestring=[s][\color{mauve}]{"}{"},
	morestring=[s][\color{black}]{>}{<},
	morecomment=[s]{<?}{?>},
	morecomment=[s][\color{dkgreen}]{<!--}{-->},
	stringstyle=\color{black},
	identifierstyle=\color{lightblue},
	keywordstyle=\color{red},
	morekeywords={material, minWidth, maxWidth, width, rotation, type, id, x, 
	y, source, target, version}% list your attributes here
}



% Copyright
%\setcopyright{none}
%\setcopyright{acmcopyright}
%\setcopyright{acmlicensed}
\setcopyright{rightsretained}
%\setcopyright{usgov}
%\setcopyright{usgovmixed}
%\setcopyright{cagov}
%\setcopyright{cagovmixed}


% DOI
\acmDOI{10.1145/nnnnnnn.nnnnnnn}

% ISBN
\acmISBN{978-x-xxxx-xxxx-x/YY/MM}

% Conference
\acmConference[GECCO '19]{the Genetic and Evolutionary Computation Conference 
2019}{July 13--17, 2019}{Prague, Czech Republic}
\acmYear{2019}
\copyrightyear{2019}

%\acmArticle{4}
\acmPrice{15.00}

% These commands are optional
%\acmBooktitle{Transactions of the ACM Woodstock conference}
%\editor{Jennifer B. Sartor}
%\editor{Theo D'Hondt}
%\editor{Wolfgang De Meuter}


\begin{document}
\title{Improved Free Form Evolution for Angry Birds Structures}

  
  \author{Laura Calle}
  \affiliation{%
    \institution{Universidad de Granada}
  }
  \email{laucalle09@gmail.com}

  \author{Juan-Juli\'an Merelo-Guerv\'os, Antonio Mora-Garc�a}
%  \orcid{1234-5678-9012}
  \affiliation{%
    \institution{Universidad de Granada/CITIC}
    \city{Granada,Spain}
    \postcode{18071}
  }
  \email{(jmerelo|amorag)@ugr.es}

  \author{Mario Garc\'ia Valdez}
  \affiliation{%
    \institution{Tecnol\'ogico Nacional de M\'exico}
    \city{Tijuana}
    \state{M\'exico}
    \postcode{22414}
  }
  \email{mario@tectijuana.edu.mx}

\renewcommand{\shortauthors}{Calle et al.}


\begin{abstract}
This paper presents an original approach based on evolutionary algorithms for 
building structures that
are stable under gravity for the physics-based puzzle game Angry
Birds, with the ultimate objective of creating
Angry Birds levels with the minimum number of constraints.
%In order to evaluate 
%the stability of the levels, they are executed in an adaptation of an 
%open source version of the game called \textit{Science Birds}. 
% Eliminated this paragraph to get some space. Also, I'm not sure
% we're doing it in this version.
We have created a custom open source evolutionary computation library,
% It is more like a library than a framework - Mario
% changed - JJ
whose main challenges have been to design a fitness function that, first, 
avoids if
possible the actual execution of the game, which is time
% that, first, avoids if possible the execution of a simulation, which is time
% I propose this change because we haven't talked about a simulator yet - Mario
consuming, and, then, to take into account the different ways in which
a structure is not sound and consider how to achieve that soundness.
%Different representations and operators have been considered and studied.
In order to test the method six experiments have been carried out,
obtaining a variety of stable structures, which is the first path for
the generation of levels that are aesthetically pleasing as well as
playable. 
% I'm not sure this is specific for this paper. Laura, can you
% clarify? - JJ
\end{abstract}

%
% The code below should be generated by the tool at
% http://dl.acm.org/ccs.cfm
% Please copy and paste the code instead of the example below. 
%


 \begin{CCSXML}
% Antonio - I have generated this CSS
<ccs2012>
<concept>
<concept_id>10010147.10010178</concept_id>
<concept_desc>Computing methodologies~Artificial intelligence</concept_desc>
<concept_significance>500</concept_significance>
</concept>
<concept>
<concept_id>10010147.10010178.10010205</concept_id>
<concept_desc>Computing methodologies~Search methodologies</concept_desc>
<concept_significance>300</concept_significance>
</concept>
<concept>
<concept_id>10010147.10010178.10010205.10010206</concept_id>
<concept_desc>Computing methodologies~Heuristic function 
construction</concept_desc>
<concept_significance>500</concept_significance>
</concept>
</ccs2012>
\end{CCSXML}

\ccsdesc[500]{Computing methodologies~Artificial intelligence}
\ccsdesc[300]{Computing methodologies~Search methodologies}
\ccsdesc[500]{Computing methodologies~Heuristic function construction}

\keywords{Search-Based Procedural Content Generator, Evolutionary algorithm, 
Game development, Angry Birds, Level generation}


\maketitle

%
%%%%%%%%%%%%%%%%%%%%%%%%%%%%%%%   INTRODUCTION   %%%%%%%%%%%%%%%%%%%%%%%%%%%%%%%
%
\section{Introduction and problem description}
\label{sec:intro}

\textit{Angry Birds} is a mobile game created by Rovio 
Entertainment 
Corporation. %\cite{angry-birds}. %, first launched in 2009. 
%The game was a huge success in the \textit{AppStore}
%---being the most downloaded mobile game in 
%history. Having been ported to several other platforms since, it has an 
%animated movie and way too many \textit{spin offs}---around seventeen.
In the game, there is a variety of defensive structures made out of
blocks which protect {\em pigs} 
from the birds fired by the player. % has to fire birds from a slingshot
%so the structure is destabilized or destroyed.
%What is interesting to us is how heavily the 
%game relies in gravity to create interesting puzzles, making the
%structures very close to reality in its dynamic behavior.
%The main challenge from 
%the Procedural Content Generation (PCG) perspective is to build stable
%structures that 
%are robust enough to take more than a single shot
%before crumbling to the ground.
%Prior to that they have to obviously
%stand on their own and not crash immediately.
%, even before a single shot has been fired.
%For the purpose of this paper, we take the definition of Procedural Game 
%Generation (PCG) presented in \cite{togelius2011procedural}.
%There is a number of competitions centered around playing AI agents for 
%specific 
%games or genres, but it is less common to find PCG competitions like the 
%\textit{Angry Birds Level Generation Competition}, whose third edition
%was celebrated in 2018 during CIG \cite{aibirds}.
%%Participants must build 
%%computer programs that are able to generate levels for the \textit{Angry 
%%Birds} game;
%Generated levels must first be valid (i. e. not crash), and then,
%they are evaluated and judged for playability and aesthetics.
%The ultimate objective of the work presented in this paper is to
%build a program capable of creating levels for \textit{Angry Birds}
%and which would be able to eventually compete in the above mentioned
%competitions. We can break it down to the following, more concise,
%objectives: 
%\begin{itemize}
%	\item Explore the expressiveness and variability of Search-Based PCG 
%	(SBPCG) using EAs.
%	\item Adapt the game code to extract data from execution to 
%	evaluate the levels.
%	\item Produce stable structures under gravity.
%	\item Place other elements on the structures to complete the levels.
%\end{itemize} 
%In this paper we will focus on the first three objectives, with sight
%on producing free-form structures that do not collapse; in a nutshell,
%we will be describing a PCG method that efficiently generates free
%form and stable Angry Birds {\em pigsty} structures.
% Antonio - I think that the paper must be better motivated. It cannot
% be just a "let's try what happens", but you should justify why this
% idea is different *and better* than other existing approaches. And
% also, what's the objective from the PCG perspective, I mean, do you
% aim to create funny levels, enjoyable ones, challenging ones, etc?
% ;)
% It's not that now, and it's not too long. 
% ***********************************************
% I have moved this to below, but we have to check we have narrative
% coherence in the introduction. - JJ
% ************************************************
%For the purpose of this paper, we take the definition of Procedural Game 
%Generation (PCG) presented in \cite{togelius2011procedural}
%as \textit{the algorithmic creation of game content with
%	limited or indirect user input} \cite{togelius2011procedural}.  
%Although PCG often uses Artificial Intelligence (AI) techniques, this
%definition does not include all uses of it in games. We do not
%consider Non Playable Character (NPC) behaviour or AI playing agents
%as content, thus they are not PCG either.
%Aesthetic elements, game rules, levels, items, stories and characters
%among others are considered content in this definition  
%\cite{togelius2011procedural}. % If this
                                % definition is not totally yours, you
                                % should give a reference for it.
                                % It is from the same  
                                %\cite{togelius2011procedural}
%Note that neither computers nor video games are mentioned in the
%definition. In fact, PCG has its roots in analog games.
%This may
%conflict with the \textit{limited or indirect user input}, but it is
%reasonable to assume that following a detailed set of
%instructions---even if it is done by a human---is not
%\textit{input}.
%The underlying concepts used by games such as \textit{Dungeon \& Dragons} 
%still 
%prevail in modern video games. \cite{smith2015analog} 
% Using an
%algorithm to assemble pre-designed pieces is a common technique in
%tabletop roleplaying guides---where the algorithm usually consists in
%several dice rolls---such as \textit{Dungeon \& Dragons}.
% It is not
%surprising that one of the early adaptations of PCG to digital
%platforms aimed to generate monsters and dungeons for physical
%games.
%
% Antonio - I think this text should be moved to the beginning of this section. 
%Or remove it.
% Moved it for now - Laura
%But the fast pace at which the game industry grows poses a challenge to 
%developers. 
%How do we create a vast amount of content that suits 
%players' expectations with low investment? PCG can tackle this by  
%increasing replayability, offering adaptive content or reducing designers' 
%workload.\cite{togelius2016introduction}
% Antonio - these two paragraphs could be moved up and should be summarized (or 
%removed). The text must be concise. ;)
% The sentence below I think is an orphan, From these factors?  define 
%replayability?
% I can do it if needed - Mario
%From these factors that we need to optimize, replayability relies on how 
%interesting is 
%playing a game more than once.
%For designers, this means their 
%product should offer more with less manually crafted content. 
%Games can also engage players by adapting gameplay elements to each individual 
%player.
%The most common way is 
%to present users with options to adjust game elements; however, by using PCG 
%techniques 
%the game itself can change based on in-game player behaviour.
%PCG can be used as a tool to assist developers. It can suggest what might be a 
%base for later development, enhancing human creativity rather than displacing 
%it.
%In this paper we target replayability by insisting in the
%generation of free-form structures.This is, that it does not follow a pattern.
%If all content generated follow a simple pattern, this is easily spotted and
%the player disengages. We aim to avoid this by creating structures whose form
%is only constrained by gravity.
In this paper we insist in the generation of free-form structures (they do not follow patterns) in attempt to enhance replayability of the game. 
% Paragraph that could be reduced to a single sentence or even
% eliminated. We are now up to 17 pages - JJ
%There are many PCG methods and it is necessary to look at some traits that 
%characterize and differentiate them from each 
%other \cite{togelius2016introduction}. The method proposed in this paper is 
%described as \textit{offline}, 
%---the PCG occurs before the game session or during development---
% \textit{necessary}, 
%---the generated content is part of the essential structure of the game---,
% \textit{generic},
%---it does not take into account player behaviour---,
%\textit{stochastic} 
%---it will not produce the same output every time it is run---,
%\textit{generate-and-test} %---it produces potentially correct solutions that 
%are tested and adjusted in each iteration before giving the actual output---
%and potentially be either \textit{automatic}
%---where the designer does not take part on the process---
%or \textit{mixed authorship}.
% where the generated content is used as a base or part of an interactive tool.
%From all these features, the actual
%The main challenge  in the game we are occupied with in this paper, Angry Birds,
%We will use Procedural Content Generation (PCG) to generate 
%potentially correct solutions, taking into account that structural integrity is not
%guaranteed by this approach. 
% ------------------------------------------------------------------

%A special kind of \textit{generate-and-test} approach to PCG is
% This was mantioned earlier so I defined the 
% acronym before. But I like it more in here.
%Search-based Procedural Content Generation (SBPCG), which is
%usually tackled with Evolutionary Algorithms(EA) \cite{togelius2010search} like 
%the 
%one proposed in \cite{hastings2009evolving}. 
%The problems faced by SBPCG are not very far from those encountered in
%EAs; since they are search
%methods, they can be a good match to perform this kind of procedures. 

%Since in Angry Birds birds are launched against structures that include pigs,
%we will be searching
%for these structures, whose playability and appearance has to be
%optimized. We try to address playability by eliminating the constraint
%that they must follow a pattern, and do not really address aesthetics
%in this paper, not even constraining the structures to be
%symmetric. However, 
%The free form implies that structural integrity is
%not guaranteed, and the structures have a realistic
%gravity, so we must avoid its collapse when they are erected at the
%beginning of a level. We will have to include this factor into the 
%\textit{fitness function}. %, which grades how good a solution is, or, as
%it might be the case, how far away it is from actually being good.
% Antonio - fitness function is part of the EA
%In Search-Based PCG, how to evaluate the quality of a solution has no straightforward 
%answer. It requires formalizing as an evaluation function how much fun, 
%exciting or engaging certain content is, which are usually based in subjective 
%assumptions. However, before being fun, the generated content must be
%valid; this is the main issue we are tackling in this paper.

%There are three main classes of fitness functions in 
%% Antonio - You should just remark the term once. ;)
%Search Based PCG \cite{togelius2010search}, but we will use a {\em direct fitness 
%function}, which takes
%measures directly on the generated content.
%In this kind of function, we extract measurable features from the generated 
%content and map them to a fitness value.

%\begin{itemize}
%	\item \textit{Direct Fitness Function} where certain easily measurable 
%	features are extracted from the generated content and mapped to a 
%	fitness value.
%	\item \textit{Simulation-Based Fitness Function} where the 
%	fitness is calculated using features from an agent's gameplay. 
%	\item \textit{Interactive Fitness Function} where the fitness value is 
%	obtained from the player, whether it be explicitly or 
%	implicitly. 
%\end{itemize}

%Since we are interested in structural integrity of the generated
%structures, 
%Neither actual gameplay nor players are taken into account. 
%We will use then a {\em direct fitness function}, which takes
%measures directly on the generated content.
%However, this is a
%time-consuming procedure since it actually involves the graphic
%representation of the structure and application of falling motion to
%its different parts. If we have to subject every single individual in
%the population to this, the size of the search space (which is huge
%since it is free form) explored is going to be very small. So we will
%have to change it to minimize the actual number of structures that are
%simulated by applying heuristics to the data structure and assigning
%it a fitness even before simulation.

%The fact that we are focusing on structures with a free form also
%implies that the data structure we will evolve has to take this into
%account. However, there is no single way to create this representation,
% so several options will have to be evaluated. 

 % Antonio - In my opinion the introduction is extremely long. It
% should be reduced, doing it more concise, focused on the paper
% objectives and with a firm motivation based on the existing gap of
% the state of the art. 
 

%The rest of this paper is organized as follows: 
%next section presents
%the state of the art in this type of level generation, as well as its
%relationship to the actual problem of generation of structurally sound
%structures. 
%the problem of generating Angry Birds levels is described
%next in Section \ref{sec:angry}. Next experiments are presented in
%section \ref{ch:res}. We present our conclusions in \ref{sec:conclusions}.


%%%%%%%%%%%%%%%%%%%%%%%%%%%%%%  STATE OF THE ART  %%%%%%%%%%%%%%%%%%%%%%%%%%%%%%
%\section{Background and State of the Art}
%\label{sec:soa}
%
%PCG is becoming more frequently used in new games as one of the main
%tools for supporting designers and developers. It has a growing
%relevance which has been also translated into the creation of new
%international competitions aimed to design algorithms to generate
%interesting/funny/enjoyable contents for different games, such as
%Super Mario (a clone of it indeed) \cite{MarioAI_Level_12}, not
%specified General Video Games
%\cite{GAIG_LevelGeneration_18,Khalifa_GVGLG_16}, or recently, for
%Angry Birds \cite{AngryBirds_LevelGeneration_18}. 

%Obviously, all the entries presented in previous editions of this last
%competition are strongly related with this
%work. Unfortunately just a few  authors have published their
%proposals in conferences or journals. The latest edition
%\cite{AngryBirds_LevelGeneration_18} was focused on finding the best 
%entry generating entertaining levels. Fun was the main factor in the
%evaluation of the participants;  creativity and difficulty were then
%also taken into account as secondary factors. 
%Six entries competed on it, being the most representative the one by
%J. Yuxuan et al., able to generate random quotations and formulae with
%the different components of a level; J. Xu et al. proposal which
%generates levels that look like pixel images; or a third approach (by
%C. Kocaogullar) that generates levels by translating music patterns to
%structures. % No reference for this? - JJ 

%The winner was a submission called Iratus Aves, an new iteration of
%the works by M. Stephenson and J. Renz
%\cite{stephenson2017generating,stephenson2016procedural}, which is a
%\textit{constructive method} where the structures displayed on the
%level are built from top to down, in several phases, recursively
%building a structure, each row composed of a single type of block
%(with a fixed rotation). The likelihood of selecting a certain block
%is given by a probability table, which was tuned using an optimization
%method. Then, the blocks are placed using a tree structure.
%% where the first selected block is the peak and blocks underneath it are split
%%into subsets that support the previous row with one, two or three
%%blocks. 
%This ensures local stability, but not global stability, which
%is tested once the whole structure is completed.  
%After that, other objects (pigs and TNT) are placed, following an
%scheme of potential positions for them according to a rank based on
%some terms such as structural protection or dispersion. 
%
%This approach was improved by adding a `layer' devoted to the
%selection of the material (stone, ice or wood) based on different
%strategies. Materials are important for the game dynamics given that certain 
%birds are more
%efficient against particular ones.
%%so they will have a high impact on
%%the game dynamics.
%Thus, a strategy sets stone to detected weak
%points using a trajectory analysis-based strategy.
%% sets to the same material
%% all blocks in the trajectory of a shot aimed at a particular pig.
%Clustering strategy takes a random block, sets its material and
%propagates to the surrounding blocks
%%that have not been assigned yet.
%Row grouping and structure blocking apply the same material to a
%whole row or structure respectively. 
%
%The main problem with this and other constructive approaches is that 
%the range of different structures created is going to be relatively small; 
%monotony
%leads to boredom, decreasing playability. On the other hand, generated
%structures are guaranteed to be structurally sound, and constructive
%approaches are generally faster than search-based procedures.
%
%\textit{Search-based approaches} propose a good alternative to deal
%with these limitations. Thus, Lucas Ferreira and Claudio Toledo
%\cite{ferreira2014search} presented a solution -- ranked 4th on the
%last competition -- based on SBPCG, which uses a Genetic Algorithm
%(GA) and a game clone named \textit{Science Birds} developed to
%evaluate the levels, the same one we use in this paper.
%
%In a GA individuals correspond to levels, each represented by an array
%of columns. Each column is a sequence of blocks, pigs and predefined
%compound blocks, using an identification integer. This representation
%also includes the distances between different columns. The population
%is initialized randomly following a probability table which defines
%the likelihood of a certain element being placed in a certain position
%inside a column. This implies that a shape is chosen beforehand, once
%again to guarantee stability, but decreases playability by generating
%structures whose only differences are which blocks are placed on top
%of which.  Levels are evaluated executing them in the simulator and
%checking their average stability, considering the speed of every block
%when erected -- which must tend to be zero for having a stable
%structure --. The authors designed specialized crossover and mutation
%operators, aiming to maintain some consistency in the new solutions
%generated.
%
%However, this paper proposes a different approach: the \textit{free
%  form evolution}. To our knowledge there are no other PCGs that
%use this approach for the generation of Angry Birds levels. 
%% Antonio - This is the main selling point of the paper, so if you
%% think this is the only approach of free form evolution in games,
%% remove the word 'many' in the sentence above. ;D
%% Done - JJ
%
%If we look outside computational intelligence in games and focus on 
%structural optimization in architecture, there are several approaches using 
%search-based
%algorithms.
%Just to cite some examples, we can find a bioinspired algorithm 
%called Cuckoo Search \cite{gandomi2013cuckoo}
%which performance was tested with structural optimization. However, this 
%optimization is heavily parametrized
%and we are looking for evolution of structures that do not follow a predefined 
%pattern, so this approach does not fit our requirements. A different take on 
%the design of structures is using Generative Grammatical 
%Encodings as Hornby and Pollack present in \cite{hornby2001advantages} where 
% l-system
%and its production rules are considered individuals. Still, this increases the
%number of patterns that can be generated, but still precludes the formation 
%of disjoint structures, for instance, a defensive tower before a simple pigsty 
%in our scope.

%Thus, we will mainly have to look outside computational intelligence in
%games to focus on optimization in architecture, where there are
%several approaches to structural optimization using search-based
%algorithms. Gandomi et al. \cite{gandomi2013cuckoo} proposed a
%bioinspired algorithm called Cuckoo Search and tested its performance
%with several structural optimization problems. The  
%algorithm is inspired by brood parasitic behaviour of some cuckoo
%species using Levy flight behaviour, a random walk that follows a certain 
%probability distribution that can be modelled mathematically.
%One of the test cases for the algorithm is a structural design of a 
%pin-jointed 
%plane frame. In this case the forces and the length of the base is predefined 
%and the frame is parametrized with two angles. The lengths of the members need 
%to be minimized but they depend on those two angles. The cuckoo search locates 
%two global optima for this problem specification. However this kind of 
%structural optimization is not flexible enough to fit our optimization
%problem, since we are looking for evolution of structures that do not
%follow any pattern.

%A different take on the design of structures is using Generative Grammatical 
%Encodings as Hornby and Pollack present in
% \cite{hornby2001advantages}. In this case, structures are formed by
%voxels and build following instructions given as commands to a
%LOGO-style turtle. They implemented an EA which evolved tables as
%structures. The commands are generated by a context-free L-system. For
%the EA, an L-system and its production rules are considered
%individuals. In order to allow these systems to evolve they are
%parametrized and constrained.

%All individuals have a predetermined number of production rules with a fixed 
%number of arguments. Each of them is then initialized with random build 
%commands grouped in blocks. The mutation of the system is done by making a 
%small change in it such as replacing one command, changing the parameter by 
%adding a constant or changing the condition equation, among others. The 
%crossover takes two individuals, copies one of them and inserts some small 
%parts of the other one in it, like a subsequence of commands for
%example. Still, this increases the number of patterns that can be
%generated, but still precludes the formation of disjoint structures, for 
%instance, a defensive tower before a simple pigsty in our scope. 

%In general, we will try to follow a realistic structure generation
%approach, without constraining it to a fixed form, thus advancing the
%state of the art by allowing the creation of Angry Birds level with
%any structure. The next section will describe how we characterize this
%problem and our approach to it.

%%%%%%%%%%%%%%%%%%%%%%%%%%%%  PROBLEM DESCRIPTION  %%%%%%%%%%%%%%%%%%%%%%%%%%%%
%\section{Problem Description}
%\label{sec:angry}

Science Birds is the main open source Angry Birds
simulator, developed by  Ferreira and Toledo
\cite{ferreira2014search}, and is available at \cite{sciencebirds}. 
Starting from there, we patched it for producing usable output 
% Antonio - I think this is important and could be considered as a contribution 
%of the work, so please, explain it and clarify.
% Better this way? - Laura
and automate its work. The modified 
version is available on \cite{sciencebirds-adapt}.
%It produces output 
%containing the position and average magnitude of the velocity of each block 
%that was not broken after the simulation.% It can be run from beginning to end 
%without user interaction and minimizes the amount of time spent on simulating 
%each level. 

%In later experiments we will use Box2D \cite{box2d}, the physics engine originally used for the actual Angry Birds game, to avoid launching
%the whole game which is time consuming.
%In this simulation we do not have the resistance of the
%blocks implemented but we can test the level stability much more efficiently
%(since there is no overhead computing things unrelated to physics). This 
%means we cannot penalize levels per broken block.

%\subsection{Evolutionary Algorithm}\label{ch:Representation}

%Once the simulator that is going to be used to measure fitness is
%ready, we have to design the fitness function.
%As obvious as it might seem,
The main feature of a stable level is that it 
is not in motion, so we will evaluate its whole stillness as 
opposed to its speed. %-- considering every single block --.
$$fitness_{ind} = \frac{1}{|V|}\sum_{i=0}^{b}{V_i} + P_{broken}\cdot(b-|V|)$$

The modified 
% Antonio - So it was not in the original simulator, was it?
simulator provides the average magnitude of
velocity for each block, % Check that this is true - JJ
% The original simulator did calculate the average but it did not offer it as 
% output and most importantly th value was frame rate dependent. So yes, it was 
% the modified version - Laura 
%This set is
a set noted as $V$. %, with $|V|$ being the cardinality of the set.
The number of blocks in an individual is $b$, used to calculate .
%It can differ from the cardinality of $V$ since broken blocks are not tracked.
the number of broken blocks  
%$b-|V|$ and it is
%multiplied by a 
and apply a penalization to invalid levels ($P_{broken} = 100$). 
% i.e. the block free-falls from a height.
%whose blocks break without user interaction would not be considered valid. This 
%happens when a block free-falls from a certain height or collides with a 
%falling object. 
%$P_{broken}$ is set to $100$ so it will separate non-valid levels from potentially good ones.
In the sixth experiment this fitness function is changed,
%after observing the results for the previous experiment, all of them 
described in section \ref{ch:res}.
%$$fitness_{indV2} = \max{(V)}$$
% When we say length of a vector, we think about the norm (the actual lenght) 
% for instance the hypotenuse in a 2D vector. I think here we are talking about 
%the
% number of elements. If V is a set, then the cardinality |V| is right. 
% From your "calculateFitness" this is len so it is a list. 
% I was confused because we are talking about magnitude, velocity, etc. 
% Consider changing the name of V to an ordered set? - Mario
% Done - Laura


%Either way, simulating a level is quite time consuming, on the order of
%seconds using the game.
% which makes it almost unfeasible for our purpose, so we
%decided to take additional factors on the fitness so that not all
%levels are actually simulated. 
%Although we may know that a given level will perform poorly in the simulation, 
%removing it from the population will cause a high loss of diversity.
%For that reason, 
%Before testing a level, there are some indicators %like
%distance to the ground and number of overlapping blocks, 
%that a level would 
%not be a valid solution
% Antonio - what do you mean? It is not clear.
%and thus simulation can be skipped, avoiding this time consuming process.
Since in-game simulation is a time consuming process, we will skip some
levels based on indicators such as 
%Levels will be penalized by 
its distance to the ground, the number of overlapping
blocks %(using Separating Axis Theorem\cite{ericson2004real})
and the number of blocks broken during the simulation. In later experiments
hight is also taken into account.
%If the lowest object is not close to the ground it is very likely that
%it, along with all the others blocks above it, 
%will break from the impact.
%Levels that have all their blocks higher than a 
%certain threshold will not be simulated in the game since they are likely to break. The threshold used is 
%$0.1$ in game units and the penalty applied to the distance in order
%to compute the fitness value is $10$.
%$$f_{distance} = 
%\begin{cases}
%P_{distance}\cdot D_{lowest}, & \text{if } D_{lowest} > threshold\\
%0, & \text{otherwise}
%\end{cases}
%$$

% This is question about your code. In there each block individual
% has a penaly attribute. This is a penalty value or is something that is
% calculated? This value is the same for all individuals? If it is maybe
% it belongs on the experiment or evolution object.
% There is a penalty that is equal for all of them. It was that way to make the 
%fitness value calculation of each individual independent. I am pretty sure it 
%would be easy to change. Does it need to be change before the paper submission 
%date? 


%The other measure is the number of overlapping blocks. The separating axis 
%theorem \cite{ericson2004real} determines if two convex shapes intersect.
%It is commonly used in game development for detecting collisions.
%A level with blocks that occupy the same space is not likely to be stable, as 
%The Unity Engine underlying the simulator will solve the issue moving the blocks until 
%there is no collision. We can not assume this process to be deterministic, since it is done by Unity proprietary code.
% it is not possible to know what it does.
% Since Unity 
% Engine is not open source and there is no documentation on how exactly those 
% collisions are solved, we assume that a precise prediction of the positions 
%of 
% the blocks is not possible and therefore the fitness value obtained could be 
% inaccurate.
%So, a penalty is applied and the level is not simulated either. 
%In this case it is $f_{overlapping} = P_{overlapping} \cdot N_{overlapping}$ 
%where the first factor is a penalty set to $10$ and the second is the number of 
%blocks that overlap with each other.

%In later experiments we will substitute the $f_{distance}$ with gap 
%in Y-axis.
%We project all blocks on the Y-axis and calculate range of values for the Y 
%coordinate
%that are not covered inside the feasible range. This is treated the same way as 
%$f_{distance}$
%(same penalization and threshold) and we call it $f_{Y-axis}$.
%$$f_{Y-axis} = 
%\begin{cases}
%P_{distance}\cdot Project_{Y-axis}, & \text{if } Project_{Y-axis} > threshold\\
%0, & \text{otherwise}
%\end{cases}
%$$

%If both $f_{distance}$ and $f_{overlapping}$ are $0$ then the level is suitable 
%for simulation and fitness is calculated as $fitness_{ind}$.
%This would be 
%considered \textit{overpenalization} but exploring unfeasible regions entails a 
%serious overhead that we need to minimize \cite{runarsson2003evolutionary}. 
%On the other hand, levels with 
%multiple blocks broken during the simulation are not feasible either but 
%running the simulation is necessary. In this case, the penalization does not 
%prevent the region to be explored.

% Paragraph candidate for elimination (back to 17 pages 
%again)--------------------------
%Previous approaches to this problem (studied in section \ref{sec:soa}) provide 
%fairly constrained outputs. The constructive method presented creates 
%pyramid-like structures and, even though there is a variety of levels, the 
%method is highly specialized in this kind of theme. The search-based approach 
%only produces tower structures adding some variety by having pre-built 
%compound 
%blocks.
% ---- That has already been mentioned in the soa section --------------

%Since one of the objectives of this work is to explore the expressiveness 
%and variability of SBPCG, it seems reasonable to use a flexible representation. 
For representation, we aim at flexible and simple structures to allow a less
directed search. %than previous solutions while keeping a 
%simple representation. 
Individuals are composed by a list of blocks 
%; platforms, TNT boxes or pigs are not considered in this 
%paper, since we are focused on the generation of structures.
%These
%building blocks 
defined by their type (shape and size), position and rotation.
%\begin{itemize}
%	\item Type: there are eight regular blocks that can be placed in the level 
%	with distinct shapes or sizes.%; they are represented as an
%        integer between $0$ and $7$.
%	\item Position: coordinates $x$ and $y$ of the centre of the block in game 
%	units.
%	\item Rotation: rotation of the block in degrees. %Here four different 
%	rotations are considered, 0, 45, 90 or 135 degrees represented as integers 
%	between $0$ and $3$.
%        \item Material: three types, which determine the durability 
%of the block. However, this does not affect their stability, so it will remain 
%constant for now as \textit{wood} material.
%\end{itemize}
%Using this representation a gene representing a single block will be formed by 
%two integers and two floating point numbers.
% Implementation detail, possibly irrelevant.
% The position of the corners of the block is frequently required, so it is 
%stored along with those attributes even though it can be calculated using the 
%size, the position and rotation of the block. (back to 16 pages with this)
%Individuals are a collection of genes, in the same way a level is a collection 
%of building blocks. 
The list is unordered and variable in length.
%The fitness of the worst individual that has been tested in game is
%stored, so that the value of not tested levels is always above ---it is a 
%minimization problem--- the in-game 
%tested levels; the starting point for fitness of such individuals is the 
%worst in-game score.
%% Implementation detail? - Laura
%The penalization is calculated using the distance of the lowest block to the 
%ground and the number of blocks that collide. 
%This requires a bit more of computation, so it will be stored and set in the 
%initialization of the individual. When a gene is modified, the number of 
%overlapping blocks is recalculated for that specific change.
%Considering all of the above, a chromosome object is composed by:
%\begin{itemize}
%	\item A list of genes.
%	\item A fitness value.
%	\item A penalty (set to {\tt False} for in-game evaluated levels).
%	\item Number of overlapping blocks (calculated).
%\end{itemize} % not sure these implementation details should be
              % included. Let's leve them, but if space is in short
              % supply, we can eliminate - JJ
%\subsection{Genetic Operators}
%Initialization is done randomly, with each individual having a random number of 
%genes, which are initialized also randomly or with a set of compound blocks like the ones found in \cite{ferreira2014search}.
%\begin{itemize}
%	\item Random: selects a random number for each attribute of the gene.
%	\item No Overlapping: also selects a random number but the gene is only 
%	added to the chromosome if it does not overlap with an already existing 
%	gene.
%	\item Discrete: selects a random number for type and rotation, but the 
%	position must be multiple of the dimensions of the smallest block (blocks 
%	will be aligned).
%	\item Discrete without overlapping: it combines the second and third 
%	initialization method.
%	\item Discrete with a set of pre configured blocks: first it includes
%	a set of blocks, and then adds blocks following the third method until
%	it reaches the desired number of blocks. The configurations used are the
%	compound blocks found in \cite{ferreira2014search}.
%\end{itemize}
%Candidates for \textit{reproduction} are selected using tournaments. 
%Two individuals are chosen from the 
%population and the best will be a parent in this generation.
%This is 
%repeated until a certain percentage of pairs have been reached. It is important 
%to note that individuals chosen are not removed from the population and 
%therefore they can appear several times in the list of parents. 
% Is there a reason for this? For small populations some individual could
% be the parent of every member of the next generation :)  - Mario
% The reason was we were getting very poor results, beacuse we start with very 
%few fairly good individuals. Should this be mentioned here? - Laura
% Yes, you should mention it. Beacuse this a good problem for future work. 
% For instance you could try to do this only at the begining and after several
% generations change it to something not as extreme. This means that the 
%parameters
% of the algorithm are dynamic or adapted. - Mario
% OK, I added it to the Future Work section. 
%Once the parents have been selected, we implement two different methods of 
%combination: sample crossover (takes a random number of genes from both parents
%without repeated blocks, produces a single individual) or a method that produces two 
%individuals, containing the blocks common to both parents and randomly distributes the 
%rest of them.
%\begin{itemize}
%	\item Sample Crossover: gives a single individual per parent pair. It takes 
%	all genes from both parents---excluding genes that are repeated---and 
%	randomly takes a number of them to create the new individual. The number of 
%	blocks is the minimum between the maximum number of blocks allowed, the 
%	mean of the two parent individuals and the number of distinct genes.
%	\item Common Blocks: produces two individuals. The common genes to both 
%	parents are passed on to both children. The remaining genes are randomly 
%	distributed to each child, half to one and half to the other. 
%	\label{ga:cross2}
%\end{itemize}
%There are four different mutations, one per gene attribute (considering position as
%two coordinates). They add or subtract a random value inside a range to the existing value of the gene. They are all applied to random members of the population.
%\begin{itemize}
%	\item Rotation: rotation is represented as an integer (it is discretized), 
%	so it adds or 
%	subtracts one 
%	to the current value. % If types follow a logical order, you
                              % should mention it. If they don't, this
                              % is the same as generating a random
                              % number - JJ
                              % it makes more sense for rotation than it does 
                              % for type actually - Laura
%	\item Type: similarly to rotation mutation.
%	\item Position X: a real value between $0$ and $1$---excluding $0$---is 
%	added or subtracted from the value of the position X.
%	\item Position Y: same as position X mutation, for position Y.
%\end{itemize}
%The new generation is selected using an elitist strategy. Best individuals in 
%both the old population and their offspring pass on the next generation, 
%maintaining the size of the population.
%************************************************
% Same as in the case of the modification, it's better to cut this
% section down to a single reference, and url of the published code -
% JJ
%\subsection{Evolutionary Algorithm Framework}\label{ch:frameworkSelection}
% The selected language for this project was Python. It has a vast Standard 
%Library and is well documented. There are a number of evolutionary programming 
%frameworks for Python. We considered two in particular for this project: 
% \begin{itemize}
% 	\item DEAP (Distributed Evolutionary Algorithms in 
%Python)\cite{fortin2012deap}
% 	\item PyEvolve\cite{perone2009pyevolve}
% \end{itemize}
% Since the game simulation will be used as a part of the fitness function, 
%there is another aspect to keep in mind. The simulation takes several seconds 
%to start, but much less between levels. When evaluating individuals in a 
%evolutionary algorithm, it would be more efficient to avoid this overhead by 
%simulating the whole population in a single execution instead of launching a 
%simulation for each individual. Then, it seems compulsory that the framework 
%used allow this behaviour.
This does not match with a binary 
representation as pure genetic algorithms suggest, so
% so the framework should be 
%flexible enough to support complex data structures.
%This prevented us from 
%using other 
%frameworks
%such as DEAP \cite{fortin2012deap}, PyEvolve \cite{perone2009pyevolve} or 
%Evolopy \cite{evolopy}
%and therefore 
a new framework was created to cover our needs (source code on 
 \cite{ab-level}).
% Say in a sentence why we are not using those above (and there's also
% Evolopy) - jj

% Other features that would be desirable for a framework to have in order to be 
%suitable for our goals are readability, ease of use and efficiency.

% Although both frameworks offer what this project is looking for, it is 
%important to note exactly how each one provides said feature:

% \begin{itemize}
% 	\item Flexible data structures: both allow them, but the user has to 
%provide the genetic operators. That is reasonable and expected. Using DEAP, 
%the 
%new data structure only needs to be registered in the \textit{Creator} class, 
%but PyEvolve requires \textit{GenomeBase}---its base module for genomes---to 
%be 
%modified in order to use custom data types.
% 	\item Customizable fitness function calls: With DEAP you can register a 
%function to evaluate individuals, but those have to take a single individual 
%as 
%a parameter. If we were to evaluate all of them at once, we would have to give 
%up the algorithms the framework provides, executing each operator and managing 
%parameters ourselves. In PyEvolve, although it is possible, it is rather 
%inconvenient since it means working around the \textit{GPopulation} class.
% \end{itemize}

% The problem tackled in this project requires high customization. Both 
%frameworks are fairly flexible and easy to use when presented with a pure 
%numerical optimization problem, but that is not the case. Taking into account 
%the time limitations for this project, productivity is really important. It is 
%not worth adapting these frameworks, considering how little we gain from their 
%use, even though it would not take a great amount of time. Instead, a new one 
%was implemented, providing all the flexibility we were looking for.

% The framework is not complex, but it covers all the features required. The 
%most important one is a fitness function that executes all the individuals at 
%once. This does not mean parallelism, in fact the steps to take cannot be 
%concurrent. First, the penalization function is applied to all individuals and 
%an XML is generated for those not penalized. Only after that, the simulation 
%is 
%run. The whole evolution process is summarized in algorithm \ref{a:evolution}.


%In order to evaluate the different options and check if they meet our
%objective, we performed a series of experiments presented in next
%section. 

%************************************************
\section{Experimentation and Results}\label{ch:res}

We set out to evolve free-form structures, but we need to test, one by
one, the different parts of our algorithm.
% fitness function, some
%evolutionary algorithm parameters and the genetic operators.
%In order to do so, 
We performed a set of initial experiments, whose results have been
published in \cite{DBLP:conf/evoW/CalleGGV19}
%whose results are shown in Table \ref{t:resOver1} and \ref{t:resOver2}.
%shows an overview of the results. 

% \begin{table}
% 	\myfloatalign
% 	\begin{tabular}{cclcl}
% 		& \textbf{Time(h)} &$\sigma$ & \textbf{G}& $\sigma$ \\ \hline
% 		\textbf{E \ref{E1}}&0.89&(0.59)&100.0&(0)\\  \hline
% 		\textbf{E \ref{E2}}&1.002&(1.97)&155.087&(240.56) \\  \hline
% 		\textbf{E \ref{E3}}&1.76&(0.6)&76.625&(42.3)\\  \hline
% 		\textbf{E \ref{E4}}&5.03&(1.46)&365.929&(158.09)  \\  \hline
% 		\textbf{E \ref{E5}}&0.27&(0.1)&894.95&(184.73)     \\  \hline 
% 		\textbf{E \ref{E6}}&0.81&(0.15)&1000.0&(0) \\  \hline
% 		% If possible, add standard deviation here - JJ
% 		% I could not find the original spreadsheet, I only have the raw data. 
% 		%I could 
% 		%%% not add standard deviation on time - Laura
% 		\hline
% 	\end{tabular}
% 	\caption{Summary of the execution of the last generation in 15-20 runs for 
% 	each experiment.
% 		G: number of generations} % What
% 	% is E? Is that the average? Number of
% 	% generations up to what?
% 	% Also explain the importance of the
% 	% fitness as an evaluation of the
% 	% quality of the solutions - JJ
% 	% Also, average over what. Best individual in every experiment?
% 	% Last generation? - JJ
% 	% Modified caption. Better? - Laura
% 	\label{t:resOver1}
% \end{table}

% \begin{table}
% 	\myfloatalign
% 	\begin{tabular}{cclclcl}
% 		& \textbf{Best}& $\sigma$  &\textbf{Avg}&$\sigma$  & 
% 		\textbf{Worst}&$\sigma$ \\ \hline
% 		\textbf{E \ref{E1}}&61.334&(133.02)&383.701&(106.14)&510.515&(133.04)\\  \hline
% 		\textbf{E \ref{E2}}&110.66&(142.21)&327.547&(238.33)&367.895&(260.83)  \\  
% 		\hline
% 		\textbf{E \ref{E3}}&0.0015&(0.003)&0.54&(0.24)&0.828&(0.34)   \\  \hline
% 		\textbf{E \ref{E4}}&0.0018&(0.003)&0.203&(0.068)&0.2997&(0.1)  \\  \hline
% 		\textbf{E \ref{E5}}& 0.065&(0.06)&0.069&(0.06)&0.071&(0.06)     \\  \hline 
% 		\textbf{E \ref{E6}}& 0.932&(0.78)&1.223&(0.74)&1.3&(0.72) \\  \hline
% 		% If possible, add standard deviation here - JJ
% 		% I could not find the original spreadsheet, I only have the raw data. 
% 		%I could 
% 		%%% not add standard deviation on time - Laura
% 		\hline
% 	\end{tabular}
% 	\caption{Summary of the results of the last generation in 15-20 runs for 
% 	each
% 		experiment.
% 		G: number of generations} % What
% 	% is E? Is that the average? Number of
% 	% generations up to what?
% 	% Also explain the importance of the
% 	% fitness as an evaluation of the
% 	% quality of the solutions - JJ
% 	% Also, average over what. Best individual in every experiment?
% 	% Last generation? - JJ
% 	% Modified caption. Better? - Laura
% 	\label{t:resOver2}
% \end{table}

%\subsection{Baseline experiment}\label{E1}

The premise of the first experiment is that our basic EA should be able to 
minimize the movement of the blocks placed on the level and
% the flexibility of 
%the representation should 
allow variety in the structures. This will
be used as a baseline.
% Antonio - Comment what kind of baseline do you need.
% This may be 
% optimistic, but we need an estimation as a starting point, and this 
%experiment 
% will serve the purpose. 
There will be four different mutations, one per gene attribute, adding or subtracting a random value inside a range to the existing value.
The crossover takes a random number of genes from both parents
without repeated blocks and produces a single individual.
The EA in this experiments uses random initialization, elitist replacement and tournament selection. The parameters are in table \ref{t:base}.
%\begin{itemize}
%	\item  Initialization with the discrete method described earlier.
%	\item  Basic sample crossover.
%	\item  All four mutations.
%	\item  Elitist replacement.
%	\item  Tournament selection.
%\end{itemize}

 % A table is better for this - JJ

% \begin{table}
% 	\myfloatalign
% 	\caption{Parameters used in the first experiment} 
% 	\label{t:base}
% 	\begin{tabular}{lclc}
% 		\hline
% 		\textbf{Population size} & 100 & \textbf{N of generations} & 100\\ \hline
% 		\textbf{\% of parents} &  0.5 &\textbf{\% of type mutations} & 0.5\\  \hline
% 		\textbf{\% of rotation mutations} & 0.5 & \textbf{\% of axis x mutations}  & 0.5 \\  \hline
% 		\textbf{\% of axis y mutations}  & 1 & &\\  \hline
% 	\end{tabular}
% \end{table}

%\begin{itemize}
%		\item \textbf{Population size}: 100 
%		\item \textbf{N of generations}: 100
%		\item \textbf{\% of parents}: 0.5 
%		\item \textbf{\% of type mutations}: 0.5
%		\item \textbf{\% of rotation mutations}: 0.5 
%		\item \textbf{\% of axis x mutations}: 0.5
%		\item \textbf{\% of axis y mutations}: 1
%\end{itemize}
% Antonio - I agree, you could save space using a table with two columns, for 
%instance. Moreover it will be easier for the reader to find the configuration 
%in a table. ;)
% Done - Laura

% The results suggest that the hypothesis was not correct. The average best 
% solution has a fitness value of 61.334 (Table \ref{t:resOver2}) 
% which indicates that probably most levels have blocks falling and  
% breaking when loaded. However, there is variety in the structures.
%The standard deviation of this measure is 133.0209 which 
%implies that while some executions performed poorly, some others may be good.
%Even the best levels have blocks that break after loading so they would not be 
%valid. However, we can tell that there is variety in the structures, since they 
%clearly differ from each other. 

%In the experiment the only termination condition was reaching the maximum 
%number of generations. Solutions found in this experiment contained just a few blocks on the ground,
%occasionally those were stacked. %as seen in figure \ref{f:e1}.
%However, looking at the results, it seems that the 
%execution ended before the population stabilized or converged.
%This means the EA may need a 
%larger number of generations to fully evolve a solution. This is the hypothesis 
%for the second experiment, which we describe next.
%Since mutation 
%percentages are high it is normal that convergence where every single 
%individual is the same one, is not reached. However, it could be possible that 
%the population is stable, where every child has a greater fitness value than 
%its parents, therefore no new members are allowed. If there are no new 
%individuals and the population is completely \textit{stuck}, the fitness value 
%of the worst individual should be the same over several generations. In Figure 
%\ref{f:grahp1} we can see that this is not the case in average. Although some 
%populations do remain the same for several generations close to the maximum, 
%most of them do not.

%\begin{figure}[H]
%	\centering
%	\includegraphics[scale=0.5]{exp1_worstIndv.png}
%	\caption{In grey, different executions of the first experiment 
%	(E1)}\label{f:grahp1}
%\end{figure}

%Most executions from the previous experiment reached the maximum number of 
%generations without stabilizing or converging.

%\begin{figure}
%	\centering
%	\includegraphics[scale=0.2]{level-0-180523_203106.png}
%	\caption{One of the results for \ref{E1}: Fitness = 9.021, G = 100  
%	}\label{f:e1}
%\end{figure}
%\subsection{Changing termination conditions} \label{E2}

In the first experiment the only termination condition was reaching the maximum 
number of generations. %Solutions found in this experiment contained just a few blocks on the ground,
%occasionally those were stacked. %as seen in figure \ref{f:e1}.
However, looking at the results, it seems that the 
execution ended before the population stabilized or converged.
The EA may need a 
larger number of generations to fully evolve a solution.
% Antonio - You must introduce the experiment (each of them), trying to justify 
%its necessity.
%The main change in this implementation is the addition 
So we will be adding two new stop 
conditions: being 10 generations without changes (stable population) or best 
fitness value below 0.01. 
% This should not be "also", it's maybe the most important change - JJ
% Antonio - I have stressed it. ;)
The set of operators is the same as the previous one, and parameters remain 
unchanged except for the \textit{Number of generations} which is equal to 1000. 
%The increase in the number of generations allows each run to fully develop 
%their individuals. This would not be feasible if we had to simulate all the 
%individuals, but thanks to the fitness function and the new stop conditions we 
%can afford this number of generations.

Although the best levels did improve respect those 
evolved in the first experiment, bad solutions have a really high fitness 
value. 
%An example of one the best levels obtained is shown in figure \ref{f:e2-4}.
%In table \ref{t:resOver2}, we can see that average fitness of levels 
%produced with this version of the EA is worse than the ones generated in the 
%first experiment.
In fact, all of the executions terminated after 10 generations with no new individuals,
%It suggest that populations can be stuck for many generations 
%before making any type of improvement.
%Any of the generated levels have a 
%fitness below 0.01 or reached maximum number of generations, which means the 
%termination criteria that stopped the evolution was that the population was 
%stable, without any new individuals added for 10 generations.
%\begin{figure}[H]
%	\centering
%	\includegraphics[scale=0.4]{exp2_explication.png}
%	\caption{Best individual evolution for all executions, grouped by number of 
%	generations}\label{f:grahp2}
%\end{figure}
%Figure \ref{f:grahp2} represents the evolution of the best individual of each 
%execution. Most of them have no more than 50 generations, therefore the 
%hypothesis for this experiment is not correct. Short evolutions show that the 
%best individual at initialization is very similar to the last one. Slight 
%improvements may be achieved by small mutations but it seems difficult for new 
%generations to outperform previous ones. We can appreciate that significant 
%improvements are most common in those executions with a poor initial 
%population. Even the ones with several hundreds of generations struggle to 
%improve the initial population.
which suggest that our hypothesis %that the number of generations could be the main factor
is not correct. It is more likely that there this EA is biased towards 
exploration rather than exploitation. 
%\begin{figure}
%	\centering
%	\includegraphics[scale=0.2]{level-0-base_large180529_223045.png}
%	\caption{One of the solutions from \ref{E2}: Fitness = 5.176, G = 767  
%	}\label{f:e2-4}
%\end{figure}
% -------------------------------------------------------------------
%\subsection{Improving exploitation using a better crossover operator}\label{E3}
As seen in the previous experiment, our genetic operators are failing to 
create new individuals that inherit good traits.
We will use a new crossover method that produces two 
individuals, containing the blocks common to both parents and randomly
distributing the rest of them.
%The second 
%crossover operator described earlier could shift the focus to exploitation.
%For the third experiment, the change introduced is in the crossover operator 
%used which was previously described as \textit{Common blocks} crossover. 
The remaining operators and termination criteria are kept the same. 

% Table \ref{t:resOver2} shows that the results have radically improved as the 
% average fitness of the best solutions drops to $0.0015$, a decrease of almost 
% 100\%. Fitness values on average and worst individual fitness suggest that the
% the population converged in most executions. 
%Additionally, it took less generations in general to reach those 
%results.
%However, executions took longer on average, which makes sense given 
%that a greater number of individuals would have been simulated. The average 
%fitness of the population and the worst individual have similar values now, 
%which suggest that in most executions the population did converge.

The levels are stable but 
% and the blocks do not fall when loading the level, but 
%it is arguable that they would be be considered structures, since
most of them consist in a 
few blocks spread about on the floor.
The average amount of blocks is $6.26$, 
which is really close to the minimum amount of blocks allowed.
%However, given 
%the proposed fitness function, it is completely logical that the evolution 
%leads to this kind of arrangements. The more objects placed on the level, the 
%more likely the individual is to not meet the requirements imposed by the 
%constraints. It also makes sense to place objects near the ground, instead of 
%one on top of the other.

%The fitness landscape a fitness function creates is difficult to
%assess, but in this case 

%\begin{figure}
%	\centering
%	\includegraphics[scale=0.2]{level-0-second_crossover180613_055622.png}
%	\caption{One of the solutions from \ref{E3}: Fitness = 1.571e-7, G = 194  
%	}\label{f:e3-4}
%\end{figure}
% -------------------------------------------------------------------
%\subsection{Changing the minimum number of blocks per individual}\label{E4}
The search goes on the direction
of minimizing the number of blocks so that the blocks
actually falling is zero.
So the only parameter that was changed for this experiment is the minimum
number of them, which went from 5 to 10. 

% The first thing to notice in these experiments (Table \ref{t:resOver1}) is the 
% increase of the average execution time.
% it went from 1.76 hours in the third 
%experiment to 5.03 hours in this one. The time spent running the simulation for 
%each population in this experiments and in the previous one should be similar. 
%However, the number of actual executions of the simulator, which only
%kick in if there is actually some block on the floor and there is no overlap, 
%drastically increased too.
%There is no doubt 
%that placing at least ten objects in a structure that does not
%collapse is more difficult than placing just five.
%The average 
%best fitness value increased slightly, while the average and worst values are 
%lower. This suggest that the latest generations of this EA are less diverse 
%than those 
%from the third experiment.
However, the results are more interesting visually. We can find some blocks 
being stacked together.
%, not only lying on the ground as in previous experiment.
%An example of the resulting structures is found in figure \ref{f:e4-1}.

% There is no reference to figures. You have to reference every single
% figure and comment it- JJ
%\begin{figure}
%	\centering
%	\includegraphics[scale=0.2]{level-0-second_crossover_min10_180602_035405.png}
%	\caption{One of the solutions from \ref{E4}: Fitness = 9.229e-7, G = 198 
%	}\label{f:e4-1}
%\end{figure}

%\subsection{Removing game and penalizing gaps in Y-axis}\label{E5}

The main problem with the previous experiments was the time needed to
load the Science Birds simulation environment and run the levels.
%which needed several seconds for loading and obtaining results.
%So the
%main objective of this experiment was to find a way to get
%rid of the in-game simulation.
In order to get rid of the in-game simulation, we will use Box2D
\cite{catto2011box2d}, %\cite{box2d}
the physics engine used in the original game.
%Since game physics do not usually resemble real world physics we adjusted
%the parameters so this simulation an the game behave in the same way.
As we can see in table \ref{t:resOver1} the execution time drastically
drops from 5 hours to less than 20 minutes on average, even running
more generations in the process.
Lower execution time allowed us to perform more operations like penalizing
not only the distance to the ground but also the \textit{gaps} in the
Y-axis.
%, which will make objects drop and maybe break.
This will encourage individuals to grow vertically and not only horizontally
like in previous experiments.
%This changes the fitness function, so we
%will have to compare by the actual obtained structure, one of which is
%shown in Figure \ref{f:e5}.
%
 \begin{figure}
 	\centering
 	\includegraphics[scale=0.3]{E5.png}
 	\caption{One of the solutions from \ref{E5}}\label{f:e5}
      \end{figure}

In general, this penalization of gaps creates a faster path to higher stable
structures.
%Still, this path leads to mostly flat structures with 
%some higher block in unstable positions, which
%are structurally solid, but not interesting.
One of the stable results is shown in figure \ref{f:e5}.
% Something more should have to be commented on these results - JJ

%\subsection{Changing the evaluation function}\label{E6}

Observing results from the previous experiment we realized that what
evolution found was that laying many blocks on the ground was enough
to get a high fitness: the average speed was decreased and it will
place unstable blocks to cover gaps in Y-axis. 
In order to correct this behaviour we changed the 
fitness function to take into account the fastest moving object. 
Additionally, we initialized levels including one of a list of pre configured
blocks (disposed as un \cite{ferreira2014search}) in addition to the random initialization used until now.
$$fitness_{indV2} = \max{(V)}$$

%This makes the fitness value depends on just one gene, although it can be a
%different one each time. The improvement of solutions to find acceptable ones
%slowed down % This would have to be proved by a chart
Again, with a different fitness function we cannot compare the fitness
value with the rest of the experiments. One of the results shown in 
Figure \ref{f:e6}.
 \begin{figure}
 	\centering
 	\includegraphics[scale=0.3]{E6.png}
 	\caption{One of the solutions from \ref{E6}}\label{f:e6}
 \end{figure}


 %Again, you can't simply show a single figure as a result. Maybe show
 %average height, or several best solutions. And always reference the
 %figure - JJ
%************************************************



%%%%%%%%%%%%%%%%%%%%%%%%%%%%%  CONCLUSIONS  %%%%%%%%%%%%%%%%%%%%%%%%%%%%
%
\section{Conclusions and Future Work} 
\label{sec:conclusions}

% ----- probably should be eliminated --- 
% Here, we briefly recap several aspects of the development of this project. 
% First, we reviewed the context of the problem: PCG. In section 
%\ref{sec:intro}, 
% we looked at the main reasons for its existence: from replacing human 
% creativity where it cannot reach (unlimited or personalized) to enhancing and 
% speeding the creative process. In some cases, PCG can  be considered a kind 
%of 
% creative expression for itself. The technical background to approach the 
% problem is discussed in section \ref{sec:soa}.
% --- up to here --

This paper was developed with
objective of exploring the expressiveness and variability of 
SBPCG with evolutionary techniques
%, adapting the game to extract data from execution 
and producing stable structures under gravity.
% What the intro to the conclusion should say are the main objectives
% of this paper, and whether they have been, or not, achieved or to
% what extent - JJ

% Antonio - This is my Proposal for this paragraph
For this aim we have implemented an Evolutionary Algorithm able to optimize 
level structures to meet this criteria. %, like the stability of the 
%constructions.
%Perhaps the level of achievement of the first objective, exploring the 
%expressiveness and variability of 
%SBPCG with evolutionary techniques,
%is not as easily measurable as the other two. It could seem like
%this objective has not been fully accomplished: only one SBPCG method
%has been implemented and tested. However, it was not in the scope of
%this work to test SBPCG in general, but in this particular case, for
%this particular game.
%Considering this, 
The method studied was
sufficiently general and flexible to draw some conclusions about the
topic. SBPCG methods are a potential good solution to offline content
generation but it requires a great amount of problem-specific
knowledge. 
%Like any other form of creative work, 
%The biggest issue may
%be how to measure how good, creative or enjoyable is the piece. 
The more rules the author adds, results tend to be 
%expressiveness starts to get lost as the results are 
variations of the same idea. However, %it is crystal clear
%from the experiments run in this paper that 
a lack of knowledge
will lead to unexpected outcomes.
In our case, the fitness function used the 
stability of the structure and only considered other features %--- overlapping 
%blocks and distance to the ground--- 
to ensure the levels would be valid.
%In experiment \ref{E5} and \ref{E6} height was also taken into account.

 % I am lost in this paragraph. It should be more
 % related to the actual experiments made - JJ

%The second objective, adapting the game to extract data from execution, was 
%certainly achieved. It was also a basic requirement to proceed with the rest of 
%them. The game does provide the data, as long as the input is correctly 
%structured. 
% Although the original intention was to make it available on Linux, 
% currently it can only be executed with the desired behaviour on Windows. 
% Antonio - I think this is not relevant
% Another issue is that the simulation is not easily adaptable. If the fitness 
% function of the EA is changed and needs data that is not included in the 
%output 
% right now, the game would have to be changed and compiled again. 
% Antonio - I would suggest this text instead
%In order to conduct our experiments the Angry Birds simulator (Science Birds) 
%has been adapted to our necessities, yielding now some  other information 
%required to evaluate the individuals of the implemented EA. As this was a 
%bottleneck, experiments \ref{E5} and \ref{E6} used only a physic simulation,
%leaving behind the actual game.
%It would be interesting to obtain other kind of data from both simulations,
%such as the height of the structure and, eventually, its resistance to
%bombardment by angry birds. However, this is left as future work.
The main issue is how we define levels and how the definition %of level 
plays
with the paths of evolution.
Producing stable structures under gravity was %the third objective and
%the closest to the ultimate one. That objective was 
effectively
achieved, but the consequence of evolving in a path of minimum movement 
%or maximum stability
results in squat
structures that are %neither aesthetically pleasing nor 
not playable,
although undeniably sturdy and stable. % and do not have more than a few floors, which could not be very 
%exciting for the players. 
% Antonio - I would add:
% , which could not be very exciting for the players. 
% Tell me what you think of this.
% Floors is not precise. ..more than a few stacked blocks? - Mario 
%The main issue is, then, how we evaluate the levels. In fact this is 
%a matter of how we define what an Angry Birds' level \textit{is} and if that 
%definition matches the fitness function.
%A lot of elements were correct, but the definition was not complete.
 % is also a problem.
%Since we define as
%level as a structure that
%If the key feature in the definition is \textit{stable}, evolution 
%will maximize stability,
%finding %, as in the beginnings of architectural practice,

%We have been successful in,
%evolving free form structures, to find these type. But once we get
%there, we need to go into a different evolution mode that takes into
%account several features as 
Next step would be treating this as a multiobjective optimization problem:
stability is the first, but we can sacrifice a bit of stability for
height or some other aesthetic quality.

% This idea brings us to the last objective, that was not achieved. Without 
% results that match the definition of an Angry Birds level, placing the 
% remaining objects would not have made the level playable. The partial 
% achievement of the third goal, blocked the fourth one as this objective was 
% dependent on the third one.

To conclude on an optimistic tone, this work provides an interesting insight 
into the SBPCG, through the completion---and failure---of the goals we
set out to achieve at the beginning.
%
%In order to improve the results of the method, different constraints could be 
%expressed as multiple objectives. Overlapping blocks and velocity could 
%be treated as minimization objectives and height as a maximization one. % but 
%of course we don't need
                                % to minimize height any more... - JJ
                                % Corrected - Laura

%If we pay attention at the stages of evolution in this work, there is also 
%room for improvement in the genetic operators. For example, the initialization 
%produces a small amount of valid individuals which suggested that an elitist 
%strategy for selection would work best. However, new experiments will help to 
%better balance exploration and exploitation. An interesting addition
%would be to add {\em building} operators that pile blocks on
%structures that are already stable. % Check out this idea, Laura - JJ
 % I like it, let's keep it - Laura

% First a discussion to say if the goal can be achieved using these
% methods. That is, finding stable structures is just a matter of
% adding more to the minimum numbre of blocks or more generations? - JJ

% ---- If we need 5 hours for every run, this is not realistic --- 
% Once we had a generator that meets our expectations, performance could be 
% enhanced by the analysis of the best set of initial parameters using Analysis 
% of Variance (or ANOVA), as presented in \cite{estevez2017statistical}. 

% ---- Some kind of "building" could be combined as genetic operator,
% but that would create structures that are too similar -- JJ
% -------------
% Some options were discarded for time limitations, so these could be main 
%areas 
% of development. The chromosome representation using generative grammatical 
% encoding\cite{hornby2001advantages}, although it would radically change the 
% structure of the method, might ensure that generated levels are consistent. 
% This would require carefully selection the operators that would be the 
%building 
% blocks of the generative grammar.


%Another important issue that needs to be addressed is the time performance. 
%Right now the simulation is the main bottleneck in the execution. One way to 
%speed up the process can be \textit{cleaning} up the current simulation, 
%getting 
%rid of any unnecessary assets while maintaining the bare minimum to
%perform evaluation. 
% However, this will not solve the problem, since the content generator will 
%still need to launch an external executable.
% Antonio - I would not mention future problems

%The communication between the simulation and the content generator is through 
%read and write operations on disk, instead of memory. This could be avoided if 
%both tools were integrated in one.
%If we aim for an online automatic generator, the generator should be 
%integrated in the game and therefore the current execution times would
%be unacceptable. However, if our goal is to 
%generate levels for mixed authorship, as an assistance to developers, it may be 
%a better idea to integrate the simulation in the generator.
% This could be done 
% by approximating in-game physics with real physics, as described in 
% \cite{blum1970stability}. % Expand on this - JJ
% Or maybe eliminate if there's no space - JJ



%----------------------- ACKNOWLEDGEMENTS -----------------------
\begin{acks}
 This paper has been supported in part by DeepBio (TIN2017-85727-C4-2-P).
\end{acks}

\bibliographystyle{ACM-Reference-Format}
\bibliography{angrybirds} 

\end{document}